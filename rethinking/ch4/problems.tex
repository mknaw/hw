\documentclass{article}
\usepackage{listings}
\usepackage{multicol}
\usepackage{amsmath}
\usepackage{amssymb}
\usepackage[a4paper,bindingoffset=0.2in,%
            left=1in,right=1in,top=1in,bottom=1in,%
            footskip=.25in]{geometry}
\usepackage{courier}


\begin{document}

\title{Chapter 5}

\section{Easy}

% 1.1
\subsection{}
In the model definition below, which line is the likelihood?
\begin{gather}
    y_i \sim \mathcal{N}(\mu, \sigma) \\
    \mu \sim \mathcal{N}(0, 10) \\
    \sigma \sim Exp(1)
\end{gather}
$y_i$ is the likelihood.

% 1.2
\subsection{}
In the model definition just above, how many parameters are in the posterior distribution? \\~\\
Two, $\mu$ and $\sigma$.

% 1.3
\subsection{}
Using the model definition above, write down the appropriate form of Bayes’ theorem
that includes the proper likelihood and priors.
\begin{gather}
    Pr(\mu, \sigma|y) = 
    \frac{\Pi_i \mathcal{N}(y|\mu, \sigma) * \mathcal{N}(\mu|0, 10) * Exp(\sigma|1)}
    {\int \int \Pi_i \mathcal{N}(y|\mu, \sigma) * \mathcal{N}(\mu|0, 10) * Exp(\sigma|1) d\mu d\sigma}
\end{gather}

% 1.4
\subsection{}
In the model definition below, which line is the linear model?
\begin{gather}
    y_i \sim \mathcal{N}(\mu, \sigma) \\
    \mu_i = \alpha + \beta x_i \\
    \alpha \sim \mathcal{N}(0, 10) \\
    \beta \sim \mathcal{N}(0, 1) \\
    \sigma \sim Exp(2)
\end{gather}
The line $\mu_i = \alpha + \beta x_i$ specifies the linear model.

% 1.5
\subsection{}
In the model definition just above, how many parameters are in the posterior distribution? \\~\\
Three - $\alpha$, $\beta$ and $\sigma$.

% 1.5
\subsection{}
Translate the quap model formula below into a mathematical model definition.
\begin{lstlisting}
    flist <- alist(
      y ~ dnorm( mu , sigma ), mu <- a + b*x,
      a ~ dnorm( 0 , 10 ),
      b ~ dunif( 0 , 1 ), sigma ~ dexp( 1 )
    )
\end{lstlisting}
\begin{gather}
    y_i \sim \mathcal{N}(\mu, \sigma) \\
    \mu_i = \alpha + \beta x_i \\
    \alpha \sim \mathcal{N}(0, 10) \\
    \beta \sim \mathcal{U}(0, 1) \\
    \sigma \sim Exp(2)
\end{gather}

% 1.5
\subsection{}
A sample of students is measures for height each year for 3 years. After the third year,
you want to fit a linear regression predicting height using year as a predictor. Write
down the mathematical model definition for this regression, using any variable names
and priors you choose. Be prepared to defend your choice of priors.

\begin{gather}
    h_i \sim \mathcal{N}(\mu, \sigma) \\
    \mu_i = \alpha + e^\beta x_i \\
    \alpha \sim \mathcal{N}(165, 30) \\
    \beta \sim \mathcal{N}(0, 10) \\
    \sigma \sim \mathcal{U}(0, 20) \\
\end{gather}
I'm assuming the question implies we are only interested in predicting year 3 growth for
these adolescents during a stage in life in which they are growing; otherwise one might
prefer an asymptotically flat curve or maybe a multi stage to account for shrinking in
late age. But, to keep it simple and in line with the material covered in the chapter,
I choose a linear regression; $\alpha$ should correspond to the mean marginal height of
the students; I'm guessing this is around 155 but I leave a wide variance to account for
the little information we have about these students. Since they are "students," I
assume they are not in the shrinking phase of life and model with log-normal $\beta$
which prevents negative growth over years, but use 0 as the mean for $\beta$ because
they may be older students, in which case they wouldn't grow that much.

% 1.5
\subsection{}
Now suppose I remind you that every student got taller each year.
Does this information lead you to change your choice of priors? How? \\~\\
I would be inclined to set a higher mean for $\beta$ than 0.

% 1.5
\subsection{}
Now suppose I tell you that the variance among heights for students of the same age is
never more than 64cm. How does this lead you to revise your priors? \\~\\
I would go for $\sigma \sim \mathcal{U}(0, 8)$, bounding $\sigma^2$ at 64.


\end{document}
EOF
