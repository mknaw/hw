\documentclass{article}
\usepackage{listings}
\usepackage{multicol}
\usepackage{amsmath}
\usepackage{amssymb}
\usepackage[a4paper,bindingoffset=0.2in,%
            left=1in,right=1in,top=1in,bottom=1in,%
            footskip=.25in]{geometry}
\usepackage{courier}

\lstset{basicstyle=\ttfamily,breaklines=true}
\lstMakeShortInline[columns=fixed]~

\begin{document}

\title{Chapter 2: Small Worlds and Large Worlds}

\section{Easy}

% 1.1
\subsection{}
Which of the expressions below correspond to the statement: \textit{the probability of rain on Monday}?
\begin{enumerate}
  \item $Pr(rain)$
  \item $Pr(rain | Monday)$
  \item $Pr(Monday | rain)$
  \item $Pr(rain, Monday) / Pr(Monday)$
\end{enumerate}
Understanding the question to mean "\textit{the probability of rain given that it is Monday,}" not "\textit{probability of Monday and rain,}" \textbf{1.} $Pr(rain|Monday)$ corresponds to the statement as well as \textbf{4.} $Pr(rain, Monday) / Pr(Monday)$.

% 1.2
\subsection{}
Which of the following statements corresponds to the expression: $Pr(Monday|rain)$?
\begin{enumerate}
  \item The probability of rain on Monday.
  \item The probability of rain, given that it is Monday.
  \item The probability that it is Monday, given that it is raining.
  \item The probability that it is Monday and that it is raining.
\end{enumerate}
\textbf{3.} is the only option that corresponds to the expression.

% 1.3
\subsection{}
Which of the expressions below correspond to the statement: \textit{the probability that it is Monday,
given that it is raining}?
\begin{enumerate}
  \item $Pr(Monday|rain)$
  \item $Pr(rain|Monday)$
  \item $Pr(rain|Monday) Pr(Monday)$
  \item $Pr(rain|Monday) Pr(Monday)/ Pr(rain)$
  \item $Pr(Monday|rain) Pr(rain)/ Pr(Monday)$
\end{enumerate}
\textbf{1.} corresponds to the statement directly, \textbf{4.} is equivalent by Bayes theorem.

% 1.4
\subsection{}
The Bayesian statistician Bruno de Finetti (1906–1985) began his book on probability theory
with the declaration: “PROBABILITY DOES NOT EXIST.” The capitals appeared in the original, so
I imagine de Finetti wanted us to shout this statement. What he meant is that probability is a device
for describing uncertainty from the perspective of an observer with limited knowledge; it has no
objective reality. Discuss the globe tossing example from the chapter, in light of this statement. What
does it mean to say “the probability of water is 0.7”? \\

The "probability" is a proxy for the actual ratio of water on the globe given our sampling technique;
while the globe can be shown to actually be 70\% covered by water, successive tosses are either water
or land, not somewhere in between.

\pagebreak
\section{Medium}

% 2.1
\subsection{}
m1.r

% 2.2
\subsection{}
m2.r

% 2.3
\subsection{}
Suppose there are two globes, one for Earth and one for Mars.
The Earth globe is 70\% covered in water.  The Mars globe is 100\% land.
Further suppose that one of these globes—you don’t know which—was tossed
in the air and produced a “land” observation.
Assume that each globe was equally likely to be tossed.
Show that the posterior probability that the globe was the Earth,
conditional on seeing “land,” $Pr(Earth|land)$, is 0.23.

\begin{gather}
  Pr(Earth|land) = \frac{Pr(land|Earth) * Pr(Earth)}{Pr(land)} \\
  Pr(Earth|land) = \frac{0.3 * 0.5}{0.5 * (1 + 0.3)} \\
  Pr(Earth|land) \simeq 0.23
\end{gather}

% 2.4
\subsection{}
Suppose you have a deck with only three cards. Each card has two sides,
and each side is either black or white. One card has two black sides.
The second card has one black and one white side. The third card has two
white sides. Now suppose all three cards are placed in a bag and shuffled.
Someone reaches into the bag and pulls out a card and places it flat on a table.
A black side is shown facing up, but you don’t know the color of the side facing
down. Show that the probability that the other side is also black is 2/3.
Use the counting method (Section 2 of the chapter) to approach this problem.
This means counting up the ways that each card could produce the observed data
(a black side facing up on the table). \\

Call the cards $BB$, $BW$, and $WW$. If the card is $WW$, we have 0 ways of having
one side be black. If the card is $BW$, we have 1 way of having one side be black.
If the card is $BB$, we have 2 ways of having one side be black. So, we have 3 total
ways of having the shown side be black, 2 of which correspond to $BB$, implying a
$2/3$ probability of the card being $BB$.

% 2.5
\subsection{}
Now suppose there are four cards: B/B, B/W, W/W, and another B/B.
Again suppose a card is drawn from the bag and a black side appears face up.
Again calculate the probability that the other side is black. \\

The number of ways for $BB_1$ and $BB_2$ to be black are both 2, for $BW$ there
is only one way. So, we have 5 total ways of drawing a black, 4 of which correspond
to $BB$ type cards, implying a $4/5$ probability of the other side being black.

% 2.6
\subsection{}
Imagine that black ink is heavy, and so cards with black sides are heavier than
cards with white sides. As a result, it’s less likely that a card with black
sides is pulled from the bag. So again assume there are three cards: B/B, B/W,
and W/W. After experimenting a number of times, you conclude that for every way
to pull the B/B card from the bag, there are 2 ways to pull the B/W card and 3
ways to pull the W/W card. Again suppose that a card is pulled and a black side
appears face up. Show that the probability the other side is black is now 0.5.
Use the counting method, as before. \\

Have $1 * n_BB + 2 * n_BW = 1 * 2 + 2 * 1 = 4$ total ways to draw a card
with a black side face up, of which only $n_BB = 2$ correspond to a black side
underneath, implying a $2/4 = 0.5$ probality.

% 2.7
\subsection{}
Assume again the original card problem, with a single card showing a black side
face up.  Before looking at the other side, we draw another card from the bag
and lay it face up on the table. The face that is shown on the new card is white.
Show that the probability that the first card, the one showing a black side, has
black on its other side is now 0.75. Use the counting method, if you can.
Hint: Treat this like the sequence of globe tosses, counting all the ways to see
each observation, for each possible first card. \\

On the first draw have 2 ways of getting $BB$, 1 way of getting $BW$. On the
second draw, have 1 way of getting $BW$, 2 ways of getting $WW$.
So, have the following numbers of ways of getting the sequences:
\begin{gather}
  BB, BW \rightarrow 2 * 1 = 2 \\
  BB, WW \rightarrow 2 * 2 = 4 \\
  BW, BW \rightarrow 1 * 0 = 0 \\
  BW, WW \rightarrow 1 * 2 = 2 \\
\end{gather}
So, 8 ways of drawing the $B, W$ sequence in total, 6 of which correspond to an
initial draw of $BB$, implying a $0.75$ probability of having a black side
underneath on the first card.

\pagebreak
\section{Hard}

% 3.1
\subsection{}
Suppose there are two species of panda bear. Both are equally common in the wild
and live in the same places. They look exactly alike and eat the same food, and
there is yet no genetic assay capable of telling them apart. They differ however
in their family sizes. Species A gives birth to twins 10\% of the time, otherwise
birthing a single infant. Species B births twins 20\% of the time, otherwise
birthing singleton infants. Assume these numbers are known with certainty, from
many years of field research.
Now suppose you are managing a captive panda breeding program. You have a new
female panda of unknown species, and she has just given birth to twins. What is
the probability that her next birth will also be twins? \\

Denote $A$, $B$ as event that panda is species $A$, $B$, $T$ as twins event.
\begin{gather}
  Pr(A|T) = \frac{Pr(T|A) * Pr(A)}{Pr(T)} \\
  Pr(A|T) = \frac{0.1 * 0.5}{0.5 * (0.1 + 0.2)} \\
  Pr(A|T) = 1/3, Pr(B|T) = 2/3
\end{gather}
So, expected probability of next litter being twins is
\begin{gather}
  1/3 * Pr(T|A) + 2/3 * Pr(T|B) = \\
  1/3 * 0.1 + 2/3 * 0.2 \simeq 16.7\%
\end{gather}

% 3.2
\subsection{}
Recall all the facts from the problem above. Now compute the probability that
the panda we have is from species A, assuming we have observed only the first
birth and that it was twins. \\

$Pr(A|T) = 1/3$ as stated before.

% 3.3
\subsection{}
Continuing on from the previous problem, suppose the same panda mother has a
second birth and that it is not twins, but a singleton infant. Compute the
posterior probability that this panda is species A. \\

After having the twins, we update our priors $P(A) = 1/3$, $Pr(B) = 2/3$.
Find the new $Pr(A|T^c)$:
\begin{gather}
  Pr(A|T^c) = \frac{Pr(T^c|A) * Pr(A)}{Pr(T^c)} \\
  Pr(A|T^c) = \frac{0.9 * 1/3}{1/3 * 0.9 + 2/3 * 0.8} \\
  Pr(A|T^c) = 36\%
\end{gather}

% 3.4
\subsection{}
A common boast of Bayesian statisticians is that Bayesian inference makes it
easy to use all of the data, even if the data are of different types.
So suppose now that a veterinarian comes along who has a new genetic test that
she claims can identify the species of our mother panda. But the test, like all
tests, is imperfect. This is the information you have about the test:
\begin{itemize}
  \item The probability it correctly identifies a species A panda is 0.8.
  \item The probability it correctly identifies a species B panda is 0.65.
\end{itemize}
The vet administers the test to your panda and tells you that the test is
positive for species A. First ignore your previous information from the births
and compute the posterior probability that your panda is species A. Then redo
your calculation, now using the birth data as well. \\

Denote the test showing species $A$ as $\alpha$, species $B$ as $\beta$.
We know that $Pr(\alpha|A) = 0.8$, $Pr(\beta|B) = 0.65$.
Let's find $Pr(A|\alpha)$ with the initial priors $Pr(A) = Pr(B) = 0.5$:
\begin{gather}
  Pr(A|\alpha) = \frac{Pr(\alpha|A) * Pr(A)}{Pr(\alpha)} \\
  Pr(A|\alpha) = \frac{0.8 * 0.5}{0.5 * 0.8 + 0.5 * 0.35} \\
  Pr(A|\alpha) \simeq 70\%
\end{gather}

Now let's recalculate $Pr(A|\alpha)$ with the new priors $Pr(A) = 0.36$,
$Pr(B) = 0.64$.
\begin{gather}
  Pr(A|\alpha) = \frac{Pr(\alpha|A) * Pr(A)}{Pr(\alpha)} \\
  Pr(A|\alpha) = \frac{0.8 * 0.36}{0.36 * 0.8 + 0.64 * 0.35} \\
  Pr(A|\alpha) \simeq 56\%
\end{gather}

\end{document}
EOF
