\documentclass{article}
\usepackage{listings}
\usepackage{multicol}
\usepackage{amsmath}
\usepackage{amssymb}
\usepackage[a4paper,bindingoffset=0.2in,%
            left=1in,right=1in,top=1in,bottom=1in,%
            footskip=.25in]{geometry}
\usepackage{courier}

\lstset{basicstyle=\ttfamily,breaklines=true}
\lstMakeShortInline[columns=fixed]~

\begin{document}
\section{Chapter 1: Building Abstractions with Procedures}

% 1.1
\subsection{}
Below is a sequence of expressions. What is the result printed by the interpreter in response to each expression? Assume that the sequence is to be evaluated in the order in which it is presented.
\begin{multicols}{2}
\begin{lstlisting}
10
10

(+ 5 3 4)
12
    
(- 9 1)
8

(/ 6 2)
3

(+ (* 2 4) (- 4 6))
6

(define a 3)
a

(define b (+ a 1))
b

(+ a b (* a b))
19
\end{lstlisting}
\columnbreak
\begin{lstlisting}
(= a b)
#f

(if (and (> b a) (< b (* a b)))
    b
    a)
4

(cond ((= a 4) 6)
      ((= b 4) (+ 6 7 a))
      (else 25))
16

(+ 2 (if (> b a) b a))
6

(* (cond ((> a b) a)
         ((< a b) b)
         (else -1))
   (+ a 1))
16
\end{lstlisting}
\end{multicols}

% 1.2
\subsection{}
Translate the following expression into prefix form:\\
\[ \frac{5 + 4 + (2 - (3 - (6 + \frac{4}{5})))}{3(6 - 2)(2 - 7)} \]
\begin{lstlisting}
(/ (+ 5 4 (- 2 (- 3 (+ 6 (/ 4 5)))))
   (* 3 (- 6 2) (- 2 7)))
\end{lstlisting}

% 1.3
\subsection{}
Define a procedure that takes three numbers as arguments and returns the sum of the squares of the two larger numbers.
\begin{lstlisting}
(define (f x y z)
    (define (sqr a) (* a a))
    (cond ((and (>= x y) (>= y z)) (+ (sqr x) (sqr y)))
          ((and (> x y) (>= z y)) (+ (sqr x) (sqr z)))
          (else (+ (sqr y) (sqr z)))))
\end{lstlisting}

\pagebreak
% 1.4
\subsection{}
Observe that our model of evaluation allows for combinations whose operators are compound expressions. Use this observation to describe the behavior of the following procedure:
\begin{lstlisting}
(define (a-plus-abs-b a b)
    ((if (> b 0) + -) a b))
\end{lstlisting}
The procedure sums $a$ with the absolute value of $b$.

% 1.5
\subsection{}
Comment on the differences between applicative and normal-order evaluation of the following:
\begin{lstlisting}
(define (p) (p))

(define (test x y)
    (if (= x 0)
    0
    y))

(test 0 (p))
\end{lstlisting}
Applicative-order will recurse into infinity as the condition's alternative, ~(p)~, evaluated. Normal order will defer calculation of ~(p)~ and instead return the condition's consequent ~0~ as ~x == 0~ is evaluated first.

% 1.6
\subsection{}
Given:
\begin{lstlisting}
(define (improve guess x)
    (average guess (/ x guess)))

(define (average x y)
    (/ (+ x y) 2))

(define (good-enough? guess x)
    (< (abs (- (square guess) x)) 0.001))
\end{lstlisting}
Comment on the implementation with ~new-if~:
\begin{lstlisting}
(define (new-if predicate then-clause else-clause)
    (cond (predicate then-clause)
    (else else-clause)))

(define (sqrt-iter guess x)
    (new-if (good-enough? guess x)
    guess
    (sqrt-iter (improve guess x) x)))
\end{lstlisting}
The special form for ~if~ does not evaluate the ~else-clause~, whereas in ~new-if~ the ~else-clause~ is evaluated in applicative-order, so the program recurses infinitely with closer and closer x values. (This resembles \textbf{1.5})

\pagebreak
% 1.7
\subsection{}
Explain why the ~good-enough?~ criterium from \textbf{1.6} is not good for computing roots for small and large numbers. Design an alternative criterium that instead checks how much ~guess~ changes from iteration to iteration.\\ \\
Hard-coding the value ~0.001~ will not be appropriate for different degrees of magnitude of the root we are trying to estimate. For example, given a radicand of 0.0001, the program will return 0 as an acceptable guess. With a very large radicand, it is possible that floating point operations necessarily "overshoot" the acceptable bound, even if our guess is the true root - that is, after squaring both, the loss of precision may make it impossible to have a difference whose absolute magnitude falls bellow the arbitrary ~0.001~. Assuming that the root-finding method converges after infinite iterations:
\begin{lstlisting}
(define (better-good-enough? previous-guess guess)
    (< (abs (/ (- previous-guess guess) guess)) 0.001))

(define (sqrt-iter guess x)
    (define (-sqrt-iter previous-guess guess x)
        (if (better-good-enough? previous-guess guess)
            guess
            (-sqrt-iter guess (improve guess x) x)))
    ; arbitrarily set to 2x on first iteration to not
    ; divide by 0 or be considered a "close" guess
    (-sqrt-iter (* 2 guess) guess x))
\end{lstlisting}

% 1.8
\subsection{}
Newton's method for cube roots is based on the fact that if $y$ is an approximation of the cube root of $x$, then a better approximation is given by the value
\[ \frac{x/y^2 + 2y}{3} \]
Implement a cube-root procedure analogous to the square root one.
\begin{lstlisting}
(define (improve guess x)
    (/ (+ (/ x (* guess guess)) (* 2 guess)) 3))

(define (qbrt-iter guess x)
    (define (-qbrt-iter previous-guess guess x)
        (if (better-good-enough? previous-guess guess)
            guess
            (-qbrt-iter guess (improve guess x) x)))
    (-qbrt-iter (* 2 guess) guess x))
\end{lstlisting}

\pagebreak
% 1.9
\subsection{}
The following two procedures defines a method for adding two positive integers in terms of the procedures ~inc~, which increments its argument by 1, and ~dec~, which decreases its argument by 1.
\begin{lstlisting}
(define (+ a b)
    (if (= a 0)
        b
        (inc (+ (dec a) b))))

(define (+ a b)
    (if (= a 0)
        b
        (+ (dec a) (inc b))))
\end{lstlisting}
Illustrate the process generated by each procedure in evaluating ~(+ 4 5)~. Are these procedures iterative or recursive?
\begin{multicols}{2}
\begin{lstlisting}
(+ 4 5)
(inc (+ 3 5))
(inc (inc (+ 2 5)))
(inc (inc (inc (+ 1 5))))
(inc (inc (inc (inc (+ 0 5)))))
(inc (inc (inc (inc 5))))
(inc (inc (inc 6)))
(inc (inc 7))
(inc 8)
9
\end{lstlisting}
\columnbreak
\begin{lstlisting}
(+ 4 5)
(+ 3 6)
(+ 2 7)
(+ 1 8)
(+ 0 9)
9
\end{lstlisting}
\end{multicols}
First is recursive, second iterative.

% 1.10
\subsection{}
The following procedure computes Ackermann's function:
\begin{lstlisting}
(define (A x y)
    (cond ((= y 0) 0)
          ((= x 0) (* 2 y))
          ((= y 1) 2)
          (else (A (- x 1)
                   (A x (- y 1))))))
\end{lstlisting}
Consider the following procedures:
\begin{lstlisting}
(define (f n) (A 0 n))
(define (g n) (A 1 n))
(define (h n) (A 2 n))
\end{lstlisting}
Give concise mathematical definitions for each.
\begin{align}
    f(n) &= 2n \\
    g(n) &= 2^n \\
    h(n) &= 2^{h(n-1)}
\end{align}

\pagebreak
% 1.11
\subsection{}
A function $f$ is defined by the rule that $f(n) = n$ if $n < 3$ and $f(n) = f(n - 1) + 2f(n - 2) + 3f(n - 3)$ if $n \geq 3$. Write a procedure that computes $f$ with by means of a recursive process. Write a procedure that computes $f$ with by means of an iterative process.
\begin{lstlisting}
(define (recur-f n)
    (if (< n 3)
        n
        (+ (recur-f (- n 1))
           (* 2 (recur-f (- n 2)))
           (* 3 (recur-f (- n 3)))
        )))

(define (iter-f n)
    (define (-iter-f a b c k)
        (if (= k 0)
            a
            (-iter-f (+ a (* 2 b) (* 3 c)) a b (- k 1))))
    (if (< n 3)
        n
        (-iter-f 2 1 0 (- n 2))))
\end{lstlisting}

% 1.12
\subsection{}
Write a procedure that computes elements of Pascal's triangle by means of a recursive process.
\begin{lstlisting}
(define (p-tri i j)
    (cond ((or (< j 0) (< i 0) (> j i)) 0)
          ((= i 0) 1)
          (else (+ (p-tri (- i 1) (- j 1))
                   (p-tri (- i 1) j)))))
\end{lstlisting}

% 1.13
\subsection{}
Prove that $Fib(n)$ is the closest integer to $\phi^n/\sqrt{5}$, where $\phi = (1 + \sqrt{5})/2$.\\ \\
Let $\psi = (1 - \sqrt{5})/2$. Show that $Fib(n) = (\phi^n - \psi^n)/\sqrt{5}$.\\
For the base case of $n = 1$, $(\phi - \psi)/\sqrt{5} = (\frac{(1 + \sqrt{5})}{2} - \frac{(1 - \sqrt{5})}{2}))/\sqrt{5} = (\frac{2\sqrt{5}}{2})/\sqrt{5} = 1 = Fib(1)$.\\
Show that $(\phi^{n+1} - \psi^{n+1})/\sqrt{5} = (\phi^n - \psi^n)/\sqrt{5} + (\phi^{n-1} - \psi^{n-1})/\sqrt{5}$.
\begin{align}
\phi^{n+1} - \psi^{n+1} &= (\phi^n - \psi^n) + (\phi^{n-1} - \psi^{n-1})\\
\phi^{n+1} - \psi^{n+1} &= (1 + 1/\phi)\phi^n - (1 + 1/\psi)\psi^n\\
(\phi - 1 - 1/\phi)\phi^n &= (\psi - 1 - 1/\psi)\psi^n\\
\phi - 1 - 1/\phi &= \frac{-1 + \sqrt{5}}{2} - \frac{2}{1 + \sqrt{5}}\\
                  &= \frac{5 - 1 - 4}{2(1 + \sqrt{5})} = 0\\
\psi - 1 - 1/\psi &= \frac{-1 - \sqrt{5}}{2} - \frac{2}{1 - \sqrt{5}}\\
                  &= \frac{5 - 1 - 4}{2(1 + \sqrt{5})} = 0\\
0(\phi^n) &= 0(\psi^n)
\end{align}
To show that $Fib(n)$ is the closest integer to $\phi^n/\sqrt{5}$, rewrite $\phi^n/\sqrt{5} = Fib(n) + \psi^n/\sqrt{5}$. If $|\psi^n/\sqrt{5}| \leq |\psi/\sqrt{5}| < \frac{1}{2}$, $Fib(n)$ is the closest integer to $\phi^n/\sqrt{5}$. Square to show $\frac{3-\sqrt{5}}{10} < \frac{1}{4}$. Since $2 < \sqrt{5} < 3$, $\frac{3-\sqrt{5}}{10} < \frac{1}{10} < \frac{1}{4}$. $\blacksquare$

% 1.14
\subsection{}
What are the orders of growth in space and the number of steps as the cent input grows in the ~count-change~ procedure?\\ \\
TODO

% 1.15
\subsection{}
The sine of an angle (specified in radians) can be computed by making use of the approximation $\sin{x} \approx x$ if $x$ is sufficiently small, and by the trigonometric identity
\[ \sin{x} = 3\sin{\frac{x}{3}} - 4\sin^3\frac{x}{3} \]
to reduce the argument of $\sin$.
Consider the following procedure for estimating $\sin{x}$:
\begin{lstlisting}
(define (cube x) (* x x x))

(define (p x) (- (* 3 x) (* 4 (cube x))))

(define (sine angle)
    (if (not (> (abs angle) 0.1))
    angle
    (p (sine (/ angle 3.0)))))
\end{lstlisting}
a. How many times is the procedure ~p~ applied when ~(sine 12.15)~ is evaluated?\\ \\
5 times ($12.15 / 3^4 > 0.1$, $12.15 / 3^5 < 0.1$).\\ \\
b. What is the order of growth in space and number of steps as a function of $a$ generated by the ~sine~ procedure when ~(sine a)~ evaluated?\\ \\
Each time $a$ triples, we need to invoke ~sine~ one more time, so the growth in number of steps of ~sine~ is $\Theta(\log_3 a) = \Theta(\log a)$. For each subsequent step, size increases constantly (another application of ~p~ in the call stack), so the growth in size is also $\Theta(\log a)$.\\ \\
More specifically, $T(a)$ is the number of steps required if it is the smallest integer s.t. $1 = \lceil(\frac{a}{0.1}) / 3^{T(a)}\rceil$. So, $1 \ge (\frac{a}{0.1}) / 3^{T(a)} \rightarrow 3^{T(a)} \ge a/0.1 \rightarrow T(a) \ge \log_3(a/0.1) = \log_3(a) - \log_3(0.1) \ge k_2\log(a)$ for some $k_2$. Since $T(a)$ is the smallest integer that fulfils the conditions, $1 < (\frac{a}{0.1}) / 3^{T(a) - 1} \rightarrow T(a) < \log_3(a/0.1) + 1 \le k_1\log(a)$ for some $k_1$. Finally, $k_1 \log(a) \ge T(a) \ge k_2 \log(a) \rightarrow T(a) = \Theta(log(a))$.
\end{document}
EOF
