
\documentclass{article}
\usepackage{amsmath}
\usepackage{amssymb}
\usepackage[a4paper,bindingoffset=0.2in,%
            left=1in,right=1in,top=1in,bottom=1in,%
            footskip=.25in]{geometry}

\begin{document}

\newcommand{\Z}{\mathbb{Z}}
\newcommand{\Mod}[1]{\ (\mathrm{mod}\ #1)}

\title{5. Linear Congruences}
\section{Exercises}

% 1
\subsection{}
Construct congruences modulo 12 with no solutions, just one solution,
and more than one solution.\\~\\
$2x \equiv 1 \Mod{12}$.\\
$5x \equiv 3 \Mod{12}$.\\
$6x \equiv 6 \Mod{12}$.

% 2
\subsection{}
Which congruences have no solutions?\\
\textbf{(a)} $3x \equiv 1 \Mod{10}$.\\
\textbf{(b)} $4x \equiv 1 \Mod{10}$.\\
\textbf{(c)} $5x \equiv 1 \Mod{10}$.\\
\textbf{(d)} $6x \equiv 1 \Mod{10}$.\\
\textbf{(e)} $7x \equiv 1 \Mod{10}$.\\~\\
\textbf{(b)}, \textbf{(c)}, \textbf{(d)}.

% 3
\subsection{}
After Exercise 2, can you guess a criterion for telling when a congruence
has no solutions?\\~\\
Such a criterion is probably $(a, m) \nmid b$.

% 4
\subsection{}
Solve\\
\textbf{(a)} $8x \equiv 1 \Mod{15}$.\\
\textbf{(b)} $9x + 10y = 11$.\\~\\
\textbf{(a)}
$8x \equiv 1 \equiv 16 \Mod{15}$. $x \equiv 2 \Mod{15}$. $x = 2$.\\
\textbf{(b)}
$9x \equiv 11 \equiv 81 \Mod{10}$. $x = 9 + 10t$.
$9 * (9 + 10t) + 10y = 11$.  $y = -7 - 9t$.

% 5
\subsection{}
Determine the number of solutions of each of the following congruences:
\begin{gather*}
    3x \equiv 6 \Mod{15}, \quad 4x \equiv 8 \Mod{15},
    \quad 5x \equiv 10 \Mod{15}\\
    6x \equiv 11 \Mod{15}, \quad 7x \equiv 14 \Mod{15}
\end{gather*}
$(3, 15) = 3$, and $3 \mid 6$, so 3 solutions.\\
$(4, 15) = 1$, and $1 \mid 8$, so 1 solution.\\
$(5, 15) = 5$, and $5 \mid 10$, so 5 solutions.\\
$(6, 15) = 3$, but $3 \nmid 11$, so no solutions.\\
$(7, 15) = 1$, and $1 \mid 14$, so 1 solution.

% 6
\subsection{}
Find all the solutions of $5x \equiv 10 \Mod{15}$.\\
We can reduce this to $x \equiv 2 \Mod{3}$, so $x = 2$.
However now must add back all the other viable $x + 3t$.
So $x \in \{2, 5, 8, 11, 14\}$.

% 7
\subsection{}
Solve the rest of the congruences in Exercise 5.\\~\\
$3x \equiv 6 \Mod{15}$ is solved by $x \in \{2, 7, 12\}$.\\
$4x \equiv 8 \Mod{15}$ is solved by $x = 2$.\\
$7x \equiv 14 \Mod{15}$ is solved by $x = 2$.

% 8
\subsection{}
Verify that 52 satisfies each of the three congruences.\\~\\
$3|52 - 1$. $5|52 - 2$. $7|52 - 3$.

\section{Problems}

% 1
\subsection{}
Solve each of the following:
\begin{gather*}
    2x \equiv 1 \Mod{17}, \quad 3x \equiv 1 \Mod{17},\\
    3x \equiv 6 \Mod{18}, \quad 40x \equiv 777 \Mod{1777}
\end{gather*}
$x = 9$.\\
$x = 6$.\\
$x \in \{2, 8, 14\}$.\\
$40x \equiv -1000 \Mod{1777}$. $x \equiv 25 \Mod{1777}$. $x = 25$.

% 2
\subsection{}
Solve each of the following:
\begin{gather*}
    2x \equiv 1 \Mod{19}, \quad 3x \equiv 1 \Mod{19},\\
    4x \equiv 6 \Mod{18}, \quad 20x \equiv 984 \Mod{1984}
\end{gather*}
$x = 10$.\\
$x = 13$.\\
$x \in \{6, 15\}$.\\
$10x \equiv 492 \equiv -500 \Mod{992}$. $x \in \{942, 1934\}$.

% 3
\subsection{}
Solve the systems\\
\textbf{(a)} $x \equiv 1 \Mod{2}, \; x \equiv 1 \Mod{3}$.\\
\textbf{(b)}
$x \equiv 3 \Mod{5}, \; x \equiv 5 \Mod{7}, \; x \equiv 7 \Mod{11}$.\\
\textbf{(c)}
$2x \equiv 1 \Mod{5}, \; 3x \equiv 2 \Mod{7}, \; 4x \equiv 3 \Mod{11}$.\\~\\
\textbf{(a)} $x = 2k_1 + 1 \equiv 1 \Mod{3}$. So $k_1 \equiv 0 \Mod{3}$.
Now write $k_1 = 3k_2$, so $x = 6k_2 + 1 \Rightarrow x \equiv 1 \Mod{6}$.\\
\textbf{(b)} $x = 5k_1 + 3 \equiv 5 \Mod{7}$.
So $5k_1 \equiv 2 \Mod{7}$, which simplifies to $k_1 \equiv 6 \Mod{7}.$
Now write $k_1 = 7k_2 + 6$, meaning $x = 35k_2 + 33$.
We know $35k_2 + 33 \equiv 7 \Mod{11}$, or $35k_2 \equiv 7 \Mod{11}$.
Solving, we get $k_2 \equiv 9 \Mod{11}$.
So can write $k_2 = 11k_3 + 9$.
Plugging back into the equation for $x$, get
$x = 385k_3 + 348$, or $x \equiv 348 \Mod{385}$.\\
\textbf{(c)}
Can write the first congruence as $x \equiv 3 \Mod{5}$, and $x$ as $x = 3 + 5k_1$.
So, plugging this into the second congruence, get
$3(3 + 5k_1) = 9 + 15k_1 \equiv 2 \Mod{7}$, or $15k_1 \equiv k_1 \equiv 0 \Mod{7}$.
So, can write $k_1 = 7k_2$ and plug back into our equation for
$x$ to get $x = 3 + 35k_2$.
Plugging this into the third congruence, have
$4(3 + 35k_2) = 12 + 140k_2 \equiv 3 \Mod{11}$, or $k_2 \equiv 3 \Mod{11}$.
So, can write $k_2 = 11k_3 + 3$ and plug back into our equation for
$x$ to get $x = 108 + 385k_3$, or $x \equiv 108 \Mod{385}$.


% 4
\subsection{}
Solve the systems\\
\textbf{(a)} $x \equiv 1 \Mod{2}, \; x \equiv 2 \Mod{3}$.\\
\textbf{(b)}
$x \equiv 2 \Mod{5}, \; 2x \equiv 3 \Mod{7}, \; 3x \equiv 4 \Mod{11}$.\\
\textbf{(c)}
$x \equiv 31 \Mod{41}, \; x \equiv 59 \Mod{26}$.\\~\\
\textbf{(a)}
From the first congruence, can write
$x$ as $x = 1 + 2k_1 \equiv 2 \Mod{3}$, meaning $k_1 \equiv 2 \Mod{3}$
So, now write $k_1 = 2 + 3k_2$ and substitute back into the equation
for $x$ to get $x = 5 + 6k_2$, or $x \equiv 5 \Mod {6}$.\\
\textbf{(b)}
From the first congruence, can write $x$ as $x = 2 + 5k_1$.
From the second, we know that
$4 + 10k_1 \equiv 3 \Mod{7}$, or $k_1 \equiv 2 \Mod{7}$.
Now write $k_1 = 2 + 7k_2$, which we plug backinto the equation for $x$
to get $x = 12 + 35k_2$.
From the third congruence, we have
$36 + 105k_2 \equiv 4 \Mod{11}$, or $k_2 \equiv 2 \Mod{11}$.
So $k_3 = 2 + 11k_2$ and $x = 82 + 385k_3$, or $x \equiv 82 \Mod{385}$.\\
\textbf{(c)}
From the first congruence, can write $x$ as
$x = 31 + 41k_1 \equiv 59 \Mod{26}$.
So $41k_1 \equiv 28 \Mod{26}$, or $x \equiv 14 \Mod{26}$.
So can write $k_1 = 14 + 26k_2$ and plug it back into the equation for $x$
to get $x = 605 + 1066k_2$, or $x \equiv 605 \Mod{1066}$.

% 5
\subsection{}
What possibilities are there for the number of solutions of a linear
congruence $\Mod{20}$?\\~\\
There can be 0, 1, 2, 5, 10 or 20 solutions.

% 6
\subsection{}
Construct linear congruences modulo 20 with no solutions, just one solution,
and more than one solution. Can you find one with 20 solutions?\\~\\
$2x \equiv 3 \Mod{20}$.\\
$3x \equiv 3 \Mod{20}$.\\
$20x \equiv 0 \Mod{20}$.

% 7
\subsection{}
Solve $9x \equiv 4 \Mod{1453}$.
\begin{align*}
    9x &\equiv -1449 \Mod{1453}\\
    x &\equiv -161 \Mod{1453}\\
    x &\equiv 1292 \Mod{1453}
\end{align*}

% 8
\subsection{}
Solve $4x \equiv 9 \Mod{1453}$.\\~\\
\begin{align*}
    2x &\equiv 731 \Mod{1453}\\
    x &\equiv 1092 \Mod{1453}
\end{align*}

% 9
\subsection{}
Solve for $x$ and $y$:\\
\textbf{(a)} $x + 2y \equiv 3 \Mod{7}, \; 3x + y \equiv 2 \Mod{7}$.\\
\textbf{(b)} $x + 2y \equiv 3 \Mod{6}, \; 3x + y \equiv 2 \Mod{6}$.\\~\\
\textbf{(a)} Write $x + 2y = 3 + 7k_1$, $3x + y = 2 + 7k_2$.
Then, subtract to get $-5x = -1 + 7(k_1 - k_2)$.
So, $5x \equiv 1 \Mod{7}$, or $x \equiv 3 \Mod{7}$.
By inspection, this means $7|y$, i.e. $y \equiv 0 \Mod{7}$.\\
\textbf{(b)} Write $x + 2y = 3 + 6k_1$, $4x + y = 2 + 6k_2$.
Subtract to get $7x = 1 - 6k_1 + 12k_2$.
$7x = 1 \Mod{6}$, or $x \equiv 1 \Mod{6}$.
Then $y \equiv 1 \Mod{6}$.

% 10
\subsection{}
Solve for $x$ and $y$:\\
\textbf{(a)} $x + 2y \equiv 3 \Mod{9}, \; 3x + y \equiv 2 \Mod{9}$.\\
\textbf{(b)} $x + 2y \equiv 3 \Mod{10}, \; 3x + y \equiv 2 \Mod{10}$.\\~\\
\textbf{(a)} Write $x + 2y = 3 + 9k_1$, $3x + y = 2 + 9k_2$.
Subtract to get $5x = 1 - 9k_1 + 18k_2$, or $5x \equiv 1 \Mod{9}$,
whose solution is $x \equiv 2 \Mod{9}$.
Then $y \equiv 5 \Mod{9}$.\\
\textbf{(b)} Write $x + 2y = 3 + 10k_1$, $3x + y = 2 + 10k_2$.
Subtract to get $5x \equiv 1 \Mod{10}$.
There are no solutions to this congruence.

% 11
\subsection{}
When the marchers in the annual Mathematics Department Parade lined up
4 abreast, there was 1 odd person; when they tried 5 in a line, there were
2 left over; and when 7 abreast, there were 3 left over.
How large is the Department?\\~\\
Let $x$ be the cardinality of the Mathematics Department.
Restating the prompt, have:
\begin{equation*}
    x \equiv 1 \Mod{4}, \quad
    x \equiv 2 \Mod{5}, \quad
    x \equiv 3 \Mod{7}
\end{equation*}
The solution to this system is:
\begin{align*}
    x &= 1 + 4k_1 \equiv 2 \Mod{5}\\
    k_1 &\equiv 4 \Mod{5}\\
    x &= 17 + 20k_2 \equiv 3 \Mod{7}\\
    k_2 &\equiv 0 \Mod{7}\\
    x &\equiv 17 \Mod{140}
\end{align*}

% 12
\subsection{}
Find a multiple of 7 that leaves the remainder 1 when divided by 2, 3, 4, 5 or 6.\\~\\
We can write number as $7x$, such that $7x \equiv 1 \Mod{k}$ for $k \in [2, 6]$.
\begin{align*}
    7x \equiv 1 \Mod{2}\\
    x \equiv 1 \Mod{2}\\
    7 + 14k_1 \equiv 1 \Mod{3}\\
    k_1 \equiv 0 \Mod{3}\\
    7 + 42k_2 \equiv 1 \Mod{4}\\
    k_2 \equiv 1 \Mod{4}\\
    49 + 168k_3 \equiv 1 \Mod{5}\\
    k_3 \equiv 4 \Mod{5}\\
    721 + 840k_3 \equiv 1 \Mod{6}\\
    840k_3 \equiv 0 \Mod{6}\\
\end{align*}
So, 721 is such a multiple of 7.

% 13
\subsection{}
Find the smallest odd $n$, $n > 3$, such that $3|n$, $5|n+2$, and $7|n+4$.\\~\\
$n$ is odd can be written as $n \equiv 1 \Mod{2}$.
Add this to the remaining conditions to get a system of congruences:
\begin{equation*}
    n \equiv 1 \Mod{2}, \quad
    n \equiv 0 \Mod{3}, \quad
    n \equiv 3 \Mod{5}, \quad
    n \equiv 3 \Mod{7}.
\end{equation*}

Solving:
\begin{align*}
    n &\equiv 1 \Mod{2}\\
    1 + 2k_1 &\equiv 0 \Mod{3}\\
    k_1 &\equiv 1 \Mod{3}\\
    3 + 6k_2 &\equiv 3 \Mod{5}\\
    k_2 &\equiv 0 \Mod{5}\\
    3 + 30k_3 &\equiv 3 \Mod{7}\\
    k_3 &\equiv 0 \Mod{7}\\
    n &= 3 + 210t
\end{align*}

The smallest odd $n > 3$ is when $t = 1$, i.e $n = 213$.\\

% 14
\subsection{}
Find the smallest integer $n$, $n > 2$, such that $2|n$, $3|n + 1$,
$4|n + 2$, $5|n + 3$ and $6|n + 4$.\\~\\
Since $6|n + 4 \Rightarrow 3|n + 1$ and $4|n + 2 \Rightarrow 2|n$,
we can remove the latter two to get a system of congruences for which
no two moduli have a greatest common divisor greater than 1:
\begin{equation*}
    n \equiv 2 \Mod{6}, \quad
    n \equiv 2 \Mod{5}, \quad
    n \equiv 2 \Mod{4}.
\end{equation*}

Solving:
\begin{align*}
    n &\equiv 2 \Mod{6}\\
    2 + 6k_1 &\equiv 2 \Mod{5}\\
    k_1 &\equiv 0 \Mod{5}\\
    2 + 30k_2 &\equiv 2 \Mod{4}\\
    1 + 15k_2 &\equiv 1 \Mod{2}\\
    k_2 &\equiv 0 \Mod{2}\\
    n &= 2 + 60t
\end{align*}

The smallest $n > 2$ is when $t = 1$, i.e $n = 62$.

% 15
\subsection{}
Find a positive integer such that half of it is a square, a third of it
is a cube, and a fifth of it is a fifth power.\\~\\
Any $n$ such $n$ must be divisible by the primes 2, 3, and 5.
Let's examine candidate $n$s made up exclusively of these factors.
Then we can write:
\begin{align*}
    n &= 2^{2x_2 + 1} * 3^{2x_3} * 5^{2x_5}\\
    n &= 2^{3y_2} * 3^{3y_3 + 1} * 5^{3y_5}\\
    n &= 2^{5z_2} * 3^{5z_3} * 5^{5z_5 + 1}
\end{align*}
From this we can derive a system of congruences for each prime's exponent.
Starting with the exponents of 2:
\begin{align*}
    2x_2 &\equiv 2 \Mod{3}\\
    x_2 &\equiv 1 \Mod{3}\\
    x_2 &= 1 + 3t\\
    2 + 6t &\equiv 4 \Mod{5}\\
    t &\equiv 2 \Mod{5}\\
    x_2 &\equiv 7 \Mod{15}
\end{align*}
Now exponents of 3:
\begin{align*}
    3y_3 &\equiv 1 \Mod{2}\\
    y_3 &\equiv 1 \Mod{2}\\
    y_3 &= 1 + 2t\\
    3 + 6t &\equiv 4 \Mod{5}\\
    t &\equiv 1 \Mod{5}\\
    y_3 &\equiv 3 \Mod{10}
\end{align*}
Now exponents of 5:
\begin{align*}
    5z_5 &\equiv 1 \Mod{2}\\
    z_5 &\equiv 1 \Mod{2}\\
    z_5 &= 1 + 2t\\
    5 + 10t &\equiv 2 \Mod{3}\\
    t &\equiv 0 \Mod{3}\\
    y_3 &\equiv 1 \Mod{6}
\end{align*}
So, can construct such an example $n$ from
$n = 2^{15} * 3^{10} * 5^{6} = 30,233,088,000,000$.

% 16
\subsection{}
The three consecutive integers 48, 49, and 50 each have a square factor.\\
\textbf{(a)} Find $n$ such that $3^2|n$, $4^2|n+1$, and $5^2|n+2$.\\
\textbf{(b)} Can you find $n$ such that $2^2|n$, $3^2|n+1$, and $4^2|n+2$?\\~\\
\textbf{(a)} We can write this as a system of congruences:
\begin{equation*}
    n \equiv 0 \Mod{9}, \quad
    n \equiv 15 \Mod{16}, \quad
    n \equiv 23 \Mod{25}
\end{equation*}
Solving:
\begin{align*}
    n &= 9k_1\\
    9k_1 &\equiv 15 \Mod{16}\\
    k_1 &\equiv 7 \Mod{16}\\
    n &= 63 + 144k_2\\
    63 + 144k_2 &\equiv 23 \Mod{25}\\
    k_2 &\equiv 15 \Mod{25}\\
    n &= 63 + 144(15 + 25t)\\
    n &\equiv 2223 \Mod{3600}
\end{align*}
\textbf{(b)} Write this as a system of congruences:
\begin{equation*}
    n \equiv 0 \Mod{4}, \quad
    n \equiv 8 \Mod{9}, \quad
    n \equiv 14 \Mod{16}
\end{equation*}
Solving:
\begin{align*}
    n &= 4k_1\\
    4k_1 &\equiv 8 \Mod{9}\\
    k_1 &\equiv 2 \Mod{9}\\
    n &= 8 + 36k_2\\
    8 + 36k_2 &\equiv 14 \Mod{16}\\
    6k_2 &\equiv 1 \Mod{16}
\end{align*}
There is no such $n$ because the congruence $6k_2 \equiv 1 \Mod{16}$ has
no solutions.

% 17
\subsection{}
If $x \equiv r \Mod{m}$ and $x \equiv s \Mod{m + 1}$, show that
\begin{equation*}
    x \equiv r(m + 1) - sm \Mod{m(m + 1)}
\end{equation*}
Similarly to previous exercises:
\begin{align*}
    x &= r + mk_1\\
    r + mk_1 &\equiv s \Mod{m + 1}\\
    mk_1 &\equiv s - r \Mod{m + 1}\\
    mk_1 &\equiv s - r + (r - s)(m + 1) \Mod{m + 1}\\
    mk_1 &\equiv m(r - s) \Mod{m + 1}\\
    k_1 &\equiv r - s \Mod{m + 1}\\
    x &= r + m(r - s + (m + 1)t)\\
    x &= r(m + 1) - sm + m(m + 1)t\\
    x &\equiv r(m + 1) - sm \Mod{m(m + 1)}\\
\end{align*}

% 18
\subsection{}
What three positive integers, upon being multiplied by 3, 5, and 7
respectively and the products divided by 20, have remainders in arithmetic
progression with common difference 1 and quotients equal to remainders?\\~\\
To begin, one can write the three numbers as follows:
\begin{align*}
    3x &= k + 20k = 21k\\
    5y &= k + 1 + 20(k + 1) = 21k + 21\\
    7z &= k + 2 + 20(k + 2) = 21k + 42
\end{align*}
From which the following congruences can be derived:
\begin{align*}
    3x &\equiv 0 \Mod{21}\\
    x &\equiv 0 \Mod{7}\\
    x &= 7t_x\\
    5y &\equiv 0 \Mod{21}\\
    y &\equiv 0 \Mod{21}\\
    y &= 21t_y\\
    7z &\equiv 0 \Mod{21}\\
    z &\equiv 0 \Mod{3}\\
    y &= 3t_z
\end{align*}
Plugging back into the first set of equations:
\begin{align*}
    t_x &= k\\
    5t_y &= k + 1\\
    t_z &= k + 2
\end{align*}
Notice that $5|k + 1$, so $k = 4$ is a good candidate.
This implies $x = 7 * 4$, $y = 21 * 1$ and $z = 3 * 6$, or
\begin{equation*}
    x = 28, \quad y = 21, \quad z = 18
\end{equation*}
As it turns out, these three numbers have the desired properties.

% 19
\subsection{}
Suppose that the moduli in the system
\begin{equation*}
    x \equiv a_i \Mod{m_i}, \quad i = 1, 2, ..., k
\end{equation*}
are not relatively prime in pairs.
Find a condition that the $a_i$ must satisfy in order that the system
have a solution.\\~\\
Recall the algorithm we employed to solve such systems of congruences:
\begin{align*}
    x &= a_1 + m_1k_1\\
    a_1 + m_1k_1 &\equiv a_2 \Mod{m_2}\\
    m_1k_1 &\equiv a_2 - a_1 \Mod{m_2}
\end{align*}
By \textbf{Theorem 1}, $k_ix \equiv a_i \Mod{m_i}$ has no solutions
if $(k_i, m_i) \nmid a_i$.
In the congruence above, if there are any $a_i$, $a_j$ such that
$(m_i, m_j) \nmid a_i - a_j$, then the system won't have a solution.
If the system does have a solution, then we'll have
$k_1 \equiv a^{*} \Mod{m_2}$, from which it follows that
$x = a_1 + m_1(a^{*} + m_2k_2) = a_1 + m_1a^{*} + m_1m_2k_2$,
or $x \equiv a_{ij} \Mod{m_im_j}$,
and we can simply restate our problem with the congruences for $m_i$, $m_j$
combined.
In other words, $(m_i, m_j)|(a_i - a_j)$ for all $i \neq j$ is such
a necessary condition.

% 20
\subsection{}
How many multiples of $b$ are there in the sequence
\begin{equation*}
    a, 2a, 3a, ..., ba \; ?
\end{equation*}
One can rewrite the series as
\begin{equation*}
    bx_1 + r_1, bx_2 + r_2, ..., bx_b + r_b
\end{equation*}
We are looking for those terms in which $r_i$ is 0, i.e.
$bx_i = ia$, implying the linear congruence $bx \equiv 0 \Mod{a}$.
From \textbf{Theorem 1}, we know there are $(a, b)$ solutions to this
congruence, corresponding to $(a, b)$ multiples of $b$ in the sequence.

\end{document}
EOF
