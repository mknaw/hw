\documentclass{article} \usepackage{amsmath}
\usepackage{amssymb}
\usepackage[a4paper,bindingoffset=0.2in,%
            left=1in,right=1in,top=1in,bottom=1in,%
            footskip=.25in]{geometry}

\begin{document}

\newcommand{\Z}{\mathbb{Z}}
\newcommand{\s}{\sigma}
\newcommand{\p}{\phi}
\newcommand{\Mod}[1]{\ (\mathrm{mod}\ #1)}

\title{10. Primitive Roots}
\section{Exercises}

% 1
\subsection{}
What are the orders of 3, 5, and 7, modulo 8?\\~\\
All are of order 2.

% 2
\subsection{}
What order can an integer have $\Mod{9}$? Find an example of each.\\~\\
$\p(9) = 6$, so orders of integers $\Mod{9}$ must be divisors of 6,
i.e. 1, 2, 3, and 6.
$1^1 \equiv 8^2 \equiv 4^3 \equiv 2^6 \equiv 1 \Mod{9}$.

% 3
\subsection{}
What is the smallest possible prime divisor of $2^{19} - 1$?\\~\\
191.

% 4
\subsection{}
Show that 3 is a primitive root of 7.
\begin{align*}
    \p(7) &= 6\\
    3 &\equiv 3 \Mod{7}\\
    3^2 &\equiv 2 \Mod{7}\\
    3^3 &\equiv 6 \Mod{7}\\
    3^4 &\equiv 4 \Mod{7}\\
    3^5 &\equiv 5 \Mod{7}\\
    3^6 &\equiv 1 \Mod{7}
\end{align*}

% 5
\subsection{}
Find, by trial, a primitive root of 10.\\~\\
3 is a primitive root of 10, as none of $3^1, 3^2, 3^3$ are $\equiv 1 \Mod{10}$,
but $3^{\p(10)} = 3^4 \equiv 1 \Mod{10}$.

% 6
\subsection{}
Which of the integers 2, 3, ..., 25 do not have primitive roots?\\~\\
8, 12, 15, 16, 20, 21, 24.

\section{Problems}

% 1
\subsection{}
Find the orders of 1, 2, ..., 12 $\Mod{13}$.\\~\\
1 (1), 12 (2), 3 (3), 6 (4), 4 (5), 12 (6), 12 (7), 4 (8), 3 (9), 6 (10),
12 (11), 2 (12).

% 2
\subsection{}
Find the orders of 1, 2, ..., 16 $\Mod{17}$.\\~\\
1 (1), 8 (2), 16 (3), 4 (4), 16 (5), 16 (6), 16 (7), 8 (8), 8 (9),
16 (10), 16 (11), 16 (12), 4 (13), 16 (14), 8 (15), 2 (16).

% 3
\subsection{}
One of the primitive roots of 19 is 2. Find all of the others.\\~\\
The numbers from 1 to 17 that are relatively prime to 18 are
1, 5, 7, 11, 13, 17.
So, the other primitive roots of 19 are $2^5 \equiv 13, 2^7 \equiv 14,
2^{11} \equiv 15, 2^{13} \equiv 3, 2^{17} \equiv 10 \Mod{19}$.

% 4
\subsection{}
One of the primitive roots of 23 is 5. Find all of the others.\\~\\
The numbers from 1 to 22 that are relatively prime to 22 are
1, 3, 5, 7, 9, 13, 15, 17, 19, 21.
So, the other primitive roots of 19 are $5^3 \equiv 10, 5^5 \equiv 20,
5^7 \equiv 17, 5^{9} \equiv 11, 5^{13} \equiv 21, 5^{15} \equiv 19,
5^{17} \equiv 15, 5^{19} \equiv 7, 5^{21} \equiv 14 \Mod{23}$.

% 5
\subsection{}
What are the orders of 2, 4, 7, 8, 11, 13, and 14 $\Mod{15}$?
Does 15 have primitive roots?\\~\\
4 (2), 2 (4), 4 (7), 4 (8), 2 (11), 4 (13), 2 (14).
15 does not have primitive roots.

% 6
\subsection{}
What are the orders of 3, 7, 9, 11, 13, 17, and 19 $\Mod{20}$?
Does 20 have primitive roots?\\~\\
4 (3), 4 (7), 2 (9), 2 (11), 4 (13), 4 (17), 2 (19).
20 does not have primitive roots.

% 7
\subsection{}
Which integers have order 6 $\Mod{31}$?\\~\\
6, 26.

% 8
\subsection{}
Which integers have order 6 $\Mod{37}$?\\~\\
11, 27.

% 9
\subsection{}
If $a, a \neq 1$, has order $t \Mod{p}$, show that
\begin{equation*}
    a^{t-1} + a^{t-2} + ... + 1 \equiv 0 \Mod{p}
\end{equation*}
Can write the sequence $a^{t-1} + a^{t-2} + ... + 1 = \frac{a^t - 1}{a - 1}$.
Since $(a - 1, p) = 1$, can multiply both sides of a congruence with
$\Mod{p}$ by $a - 1$ to leave $a^t - 1$.
Since by defintion $a^t \equiv 1 \Mod{p}$, then $a^t - 1 \equiv 0 \Mod{p}$.

% 10
\subsection{}
If $g$ and $h$ are primitive roots of an odd prime $p$, then $g \equiv h^k \Mod{p}$
for some integer $k$. Show that $k$ is odd.\\
Proved in the text that if $h$ is a primitive root of $p$, then the least residue
of $h^k$ is a primitive root of $p$ if and only if $(k, p - 1) = 1$.
Also know that $2|p - 1$, since $p$ is an odd prime.
If also had $2|k$, then $(k, p - 1)$ would be at least 2.

% 11
\subsection{}
Show that if $g$ and $h$ are the primitive roots of an odd prime $p$,
then the last residue of $gh$ is not a primitive root of $p$.\\~\\
Know that $g \equiv h^k \Mod{p}$ with an odd $k$ from previous exercise.
Then $gh \equiv h^{k + 1} \Mod{p}$.
Since $k + 1$ and $p - 1$ are both even, then $(k + 1, p - 1) \neq 1$,
so $gh$ cannot be a primitive root of $p$.

% 12
\subsection{}
If $g$, $h$, and $k$ are primitive roots of $p$, is the least residue of
$ghk$ always a primitive root of $p$?\\~\\
Again write $g \equiv k^m$, $h \equiv k^n$, and
$ghk \equiv k^{m + n + 1}$, with $m + n + 1$ odd.\\
Can one have $(m + n + 1, p - 1) \neq 1$?

% 13
\subsection{}
Show that if $a$ has order 3 $\Mod{p}$, then $a + 1$ has order 6 $\Mod{p}$.\\~\\
$a^3 - 1 = (a - 1)(a^2 + a + 1) \equiv 0 \Mod{p}$.
Notice that $(a + 1)^3 = a^3 + 3a^2 + 3a + 1 \equiv 3(a^2 + a + 1) - 1 \equiv -1
\Mod{p}$.
So $(a + 1)^6 \equiv 1 \Mod{p}$.
So, 6 must be a multiple of the order of $a + 1$.
We already showed it can't be 3.
For 1, $a + 1 \equiv 2 \Mod{p}$.
For 2, $(a + 1)^2 = a^2 + 2a + 1 = a^2 + a + 1 + a \equiv a \not\equiv 1 \Mod{p}$.

% 14
\subsection{}
If $p$ and $q$ are odd primes and $q \mid a^p + 1$,
show that either $q \mid a + 1$ or $q = 2kp + 1$ for some integer $k$.\\~\\
If $q|a^p + 1$, can write $a^p \equiv -1 \Mod{q}$.
This means that $a^{2p} \equiv 1 \Mod{q}$, and the order of $a$
must divide $2p$.
We can eliminate 1, because if $a \equiv 1 \Mod{q}$, then $a^p \equiv 1 \Mod{q}$,
which was directly contradicted above.
For the same reason, we can eliminate an order of $p$.
So, $a$ can have either an order of 2 or $2p$.
If it's 2, then $a^2 \equiv 1 \Mod{q}$, so have $rq = a^2 - 1 = (a + 1)(a - 1)$.
Since we know that $a \not\equiv 1 \Mod{q}$, that means $q|a + 1$.
If it is $2p$, then $2p|\p(q)$, i.e. $2p|q - 1$. Then $q = 2kp + 1$.

% 15
\subsection{}
Suppose that $a$ has order 4 $\Mod{p}$.
What is the least residue of $(a + 1)^4 \Mod{p}$?\\~\\
Know that $a^2 \equiv -1 \Mod{p}$ - can see this by writing
$kp = (a^2 + 1)(a^2 - 1)$ and observing that $p \nmid a^2 - 1$
(or the order of $a$ would be 2, not 4), so $p|a^2 - 1$.
Expanding, $(a + 1)^4 = a^4 + 4a^3 + 6a^2 + 4a^1 + 1 \equiv
1 - 4a - 6 + 4a + 1 \equiv -4 \Mod{p}$.

% 16
\subsection{}
Show that $131071 = 2^{17} - 1$ is prime.\\~\\
By the corollary from the text, any divisor of $2^p - 1$ must be of the form
$2*17*k + 1$.
If 131071 is composite, it will have a divisor that is
$\leq \sqrt{131071} \approx 360$.
Furthermore, one of its divisors must be prime.
That leaves us with candidates: 103, 137, 239, 307.
None of these divides 131071.

% 17
\subsection{}
Show that $(2^{19} + 1)/3$ is prime.\\~\\
By extension of the prompt of exercise 14, for a number to divide
$2^p + 1$, it must either be equal to 3 or must have the form $2kp + 1$.
Presumably, we know 3 divides $3|2^{19} + 1$ if we believe that
$(2^{19} + 1)/3 \in \Z$.
That leaves us with candidates of the form $2 * 19 * k + 1 = 38 * k + 1$.
By a similar logic as before, one of these must be of the form $3q$,
where $q$ is a prime.

% 18
\subsection{}
If $g$ is a primitive root of $p$, show that two consecutive powers of $g$
have consecutive least residues. That is, show that there exists $k$ such that
$g^{k + 1} \equiv g^k + 1 \Mod{p}$.\\~\\
Rewrite the congruence as $g^k(g - 1) \equiv 1 \Mod{p}$.
Since $g$ is a primitive root of $p$, there exists a $k$ such that
$g^k = x$ for any $x$ in 1, 2, 3, ..., $p - 1$.
Also know that $g - 1$ can only be $\equiv 0 \Mod{p}$ if $\p(p) = p - 1 = 1$,
i.e. $p = 2$, but 2 has no primitive roots, so we can exclude it.
Furthermore, since $(g - 1, p) = 1$, we know that $x(g - 1) \equiv 1 \Mod{p}$
must have exactly one solution, and that solution must be in $1, ..., p - 1$.

% 19
\subsection{}
If $g$ is a primitive root of $p$, show that no three consecutive powers of $g$
have consecutive least residues. That is, show that
$g^{k+2} \equiv g^{k+1} + 1 \equiv g^k + 2 \Mod{p}$ is impossible for any $k$.\\~\\
Notice that the last two parts of the congruence boil down to
$g^k(g - 1) \equiv 1 \Mod{p}$, as in the previous exercise.
Now consider only the first and third parts of the congruence,
$g^{k+2} \equiv g^k + 2 \Mod{p}$.
This can be rewritten as $g^k(g^2 - 1) = g^k(g + 1)(g - 1) \equiv 2 \Mod{p}$.
Since $g^k(g - 1) \equiv 1 \Mod{p}$ must hold for the whole thing to hold,
the last congruence would require $g + 1 \equiv 2$, or $g \equiv 1 \Mod{p}$.
But, since $g$ is a primitive root of $p$, no power of $g$ lesser than
$g^{p - 1}$ is $\equiv 1 \Mod{p}$.

% 20
\subsection{}
\textbf{(a)} Show that if $m$ is a number having primitive roots, then the
product of the positive integers less than or equal to $m$ and relatively
prime to it is congruent to -1 $\Mod{m}$.\\
\textbf{(b)} Show that the result in \textbf{(a)} is not always true if $m$
does not have primitive roots.\\~\\
\textbf{(a)} Know from the text that numbers with primitive roots
must be of the form $1, 2, 4, p^e$, and $2p^e$.\\
For 1, the statement holds, because any digit $+ 1$ is divisible by 1.\\
For 2, the statement holds because $1 \equiv -1 \Mod{2}$.\\
For 4, the statement holds because $1*3 \equiv -1 \Mod{4}$.\\
For $p$, every integer from 1 to $p - 1$ is relatively prime to $p$.
We can thus write the product as $(p - 1)!$, which we know from Wilson's
Theorem is $\equiv -1 \Mod{p}$.\\
Now consider $p^e$ with $e > 1$.
For even $p$, i.e. $p = 2$, the product must consist of only odd integers,
so adding 1 to whatever this product is will produce an even number.
Now handle odd $p$.
The product of the integers up to $p - 1$ is the familiar $(p - 1)!$.
Then we have sequences of integers from $kp + 1$ to $(k + 1)p - 1$.
Notice that $\Mod{p}$, this is the same as a sequence of 1 to $p - 1$.
We have $p^{e - 1}$ such sequences, which we've shown are all individually
$\equiv -1 \Mod{p}$.
Since $p$ odd, $p^{e - 1}*-1 \equiv -1 \Mod{p}$.\\
For $2p$, the product of integers must exclude all even ones, and thus looks
like $1*3*...*(p - 1)*(p + 2)*(p + 4)*...*(2p - 1)$.
If you take the $p + 2k$ terms $\Mod{p}$, these "fill in" the dropped even
terms of the product up to $p$, so we're left with $(p - 1) \equiv -1 \Mod{p}$.\\ 
For $2p^e$, the argument is the same as for $p^e$ with odd $p$.\\
\textbf{(b)} The product for $8$ is $1 * 3 * 5 * 7 = 105$,
which is $\equiv 1 \Mod{8}$.

\end{document}
EOF
