
\documentclass{article}
\usepackage{amsmath}
\usepackage{amssymb}
\usepackage[a4paper,bindingoffset=0.2in,%
            left=1in,right=1in,top=1in,bottom=1in,%
            footskip=.25in]{geometry}

\begin{document}

\newcommand{\Z}{\mathbb{Z}}

\title{3. Linear Diophantine Equations}
\section{Exercises}

% 1
\subsection{}
Why does $2x + 4y = 5$ have no integer solutions?\\~\\
The smallest possible $x, y$ solution is $1, 1$.
$2*1 + 4*1 = 6 > 5$.
Any greater $x, y$ will produce an integer greater than $5$.

% 2
\subsection{}
Find by inspection a solution of $x + 5y = 10$ and use it to write five
other solutions.\\~\\
$x = 5, y = 1$ is such a solution.
5 others are $(15, -1), (20, -2), (25, -3), (30, -4), (35, -5)$.

% 3
\subsection{}
Which of the following linear diophantine equations is impossible?\\
\textbf{(a)} $14x + 34y = 90$.\\
\textbf{(b)} $14x + 35y = 91$.\\
\textbf{(c)} $14x + 36y = 93$.\\~\\
\textbf{(c)} is the only one for which $(a, b) \nmid c$.

% 4
\subsection{}
Find all solutions of $2x + 6y = 20$.\\~\\
This is the same as solutions of $x + 3y = 10$.
By inspection, we have $x = 7, y = 1$ as a solution.
So, the solutions are $x = 7 + 3t$ and $y = 1 - t$.

% 5
\subsection{}
Find all the solutions of $2x + 6y = 18$ in \textit{positive} integers.\\~\\
Solutions of the equation are $x = 9 + 3t, y = -t$.
To have $x, y \in \Z^+$, we must have $-3 < t < 0$, i.e. $t \in \{-2, -1\}$.
So, $x = 3, y = 2$, $x = 6, y = 1$ are all the positive integer solutions.


\section{Problems}

% 1
\subsection{}
Find all the integer solutions of $x + y = 2$, $3x - 4y = 5$, and $15x + 16y = 17$.\\~\\
$x = 1 + t, y = 1 - t$.\\
$x = 3 - 4t, y = 1 - 3t$.\\
$x = -17 + 16t, y = 17 - 15t$.

% 2
\subsection{}
Find all the integer solutions of $2x + y = 2$, $3x - 4y = 0$, and $15x + 18y = 17$.\\~\\
$x = 1 + t, y = -2t$.\\
$x = 4t, y = 3t$.\\
There are none, since $(18, 15) = 3, 3 \nmid 17$.

% 3
\subsection{}
Find all the integer solutions in positive integers of $x + y = 2$,
$3x - 4y = 5$, and $6x + 15y = 51$.\\~\\
$x = 1 + t, y = 1 - t$. These are positive for $t = 0$, so $x = 1, y = 1$.\\
$x = 3 - 4t, y = 1 - 3t$. These are positive for all $t \geq 0$.\\
$x = 1 + 5t, y = 3 - 2t$. These are positive for $t \in \{0, 1\}$,
so $x = 1, 6$, $y = 3, 1$ are the only solutions $\ni x, y \in \Z^+$.

% 4
\subsection{}
Find all the integer solutions in positive integers of $2x + y = 2$,
$3x - 4y = 0$, and $7x + 15y = 51$.\\~\\
$x = 1 + t, y = -2t$. These are not both positive for any $t$.\\
$x = 4t, y = 3t$. These are positive for all $t > 0$.\\
$x = 3 + 15t, y = 2 - 7t$. These are positive only when $t = 0$, i.e. $x = 3, y = 2$.

% 5
\subsection{}
Find all the positive solutions in integers of
\begin{align}
    x + y + z &= 31,\\
    x + 2y + 3z &= 41
\end{align}

Subtracting the two equations, have $2x + y = 52$.
Positive solutions for this equation are $x = 25 + t, y = 2 - 2t$.
Using these in the first equation, $(25 + t) + (2 - 2t) + z = 31$,
implying $z = 4 + t$.
$x, y, z \in \Z^+$ when $t \in [-3, 0]$.
So, positive solutions to the equation are
$x = 22, 23, 24, 25, y = 8, 6, 4, 2, z = 1, 2, 3, 4$.

% 6
\subsection{}
Find the five different ways a collection of 100 coins - pennies, dimes, and
quarters- can be worth exactly \$4.99.\\~\\
The solutions to this must satisfy
\begin{align}
    p + d + q &= 100,\\
    p + 10d + 25q &= 499
\end{align}
Subtracting the two, have $3d + 8q = 133$.
By inspection have a solution to this when $d = 15 + 8t$, $q = 11 - 3t$.
Using the first equation, $p + (15 + 8t) + (11 - 3t) = 100$.
So, must have p in the form of $74 - 5t$.
$p, d, q \in \Z^+$ when $t \in [-1, 3]$.
This is true for $p = 79, 74, 69, 64, 59, d = 7, 15, 23, 31, 39, q = 14, 11, 8, 5, 2$.


% 7
\subsection{}
A man bought a dozen pieces of fruit-apples and oranges-for 99 cents.
If an apple costs 3 cents more than an orange, and he bought more apples
than oranges, how many of each did he buy?\\~\\
The solutions to this must satisfy
\begin{align}
    a + o &= 12\\
    (x + 3)a + xo &= 99
\end{align}
Rewriting the second equation, have $xa + xo + 3a = 12x + 3a = 99 \Rightarrow a = 33 - 4x$.
We know that the fruit probably is not free, so $x > 0$, and the number of apples
purchased is must be at least 7.
So feasible values for $x$ must be between 1 and 6.
We also can plug this into the firust equation to get $33 - 4x + o = 12$,
which can be rewritten as $4x - o = 21$.
Since the number of oranges purchased is also non-negative, we can reject all $x$
such that $4x \leq 21$, that is, all $x \in [1, 5]$,
leaving us with viable $x = 6$. So, 9 apples and 3 oranges were purchased.

% 8
\subsection{}
The enrollment in a number theory class consists of sophomores, juniors, and
backwards seniors. If each sophomore contributes \$1.25, each junior contributes
\$.90 and each senior \$.50, the instructor will have a fund of \$25.
There are 26 students; how many of each?\\~\\
The solutions to this must satisfy
\begin{align}
    x + y + z &= 26\\
    125x + 90y + 50z &= 2500
\end{align}
Where $x, y, z$ denote sophomores, juniors, and seniors, respectively.
We can rewrite the bottom equation as $25x + 18y + 10z = 500$.
Subtracting the two equations, have $15x + 8y = 240$.
This has solutions in the form $x = 16 + 8t$, $y = -15t$.
Since the $x, y \in \Z^+$, must have $t = -1$.
So, we know $x = 8$, $y = 15$.
From the first equation we can get $z = 3$.

% 9
\subsection{}
The following problem first appeared in an Indian book written about 850 AD.
Three merchants found a purse along the way. One of them said, "If I secure
this purse, I shall become twice as rich as both of you with your money on
hand." Then the second said, "I shall become thrice as rich as both of you."
The third man said, "I shall become five times as rich as both of you."
How much did each merchant have, and how much was in the purse?\\~\\
Denoting the money in the purse as $p$ and the capital of each merchant
as $x, y, z$, the solutions must satisfy:
\begin{align}
    x + p = 2(y + z)\\
    y + p = 3(x + z)\\
    z + p = 5(x + y)
\end{align}
Subtracting the first two equations, have $x - y = -3x + 2y - z$,
which in turn can be written as $z = -4x + 3y$.
Can plug this into the third equation to get $p = 9x + 2y$.
So now have
\begin{align}
    5x = z\\
    3x = y\\
    p = 15x
\end{align}
So if $x = 1$, then $y = 3$, $z = 5$, and $p = 15$.
But we could also have $p = 30$, $x = 2$, $y = 6$, and $z = 10$.
In fact can set any $x \in \Z^+$ to solve.

% 10
\subsection{}
A man cashes a check for $d$ dollars and $c$ cents at a bank.
Assume that the teller by mistake gives the man $c$ dollars and $d$ cents.
Assume that the man does not notice the error until he has spent 23 cents.
Assume further that he then notices that he has $2d$ dollars and $2c$ cents.
Assume still further that he asks you what amount the check was for.
Assuming that you can accept all the assumptions, what is the answer?\\~\\

The man was given some amount (in cents) that can be written as $100c + d$.
Furthermore, we know $100c + d - 23 = 100(2d) + 2c$.
Rewriting, have $98c - 199d = 23$.
A solution to this is $c = 51, d = 25$.
So, the original amount given at the bank was \$25.51.

\end{document}
EOF
