\documentclass{article}
\usepackage{amsmath}
\usepackage{amssymb}
\usepackage[a4paper,bindingoffset=0.2in,%
            left=1in,right=1in,top=1in,bottom=1in,%
            footskip=.25in]{geometry}

\begin{document}

\newcommand{\Z}{\mathbb{Z}}

\title{2. Unique Factorization}
\section{Exercises}

% 1
\subsection{}
How many even primes are there? How many whose last digit is 5?\\~\\
2 is the only even prime; any other even number is divisible by 2, and thus not prime.
5 is the only number whose last digit is 5; any other number in base 10 whose last
digit is 5 is divisible by 5.

% 2
\subsection{}
Prove by induction: $\forall n \in \Z^+$, $n$ can be written as a product of primes.\\~\\
We know this holds for $n = 1$ and $n = 2$, both of which are simple products of 1 and a prime.
Assume now that the property holds for $n \leq k$.
If $k + 1$ is prime, then it is clearly a product of 1 and a prime (like 1 and 2).
Otherwise, $k + 1 = ab$, with $a, b \leq k$.
But, by the inductive assumption, $a$ and $b$ can be written as products of primes,
so $k + 1$ is itself a product of primes.

% 3
\subsection{}
Write prime decompositions for 72 and 480.\\~\\
$72 = 2 * 2 * 2 * 3 * 3 = 2^3 * 3^2$.\\
$480 = 2 * 2 * 2 * 2 * 2 * 3 * 5 = 2^5 * 3 * 5$.

% 4
\subsection{}
Which members of the set less than 100 are not prome?\\~\\
All members of the set less than 100 are as follows:\\
1, 5, 9, 13, 17, 21, 25, 29, 33, 37, 41, 45, 49, 53, 57, 61, 65, 69, 73, 77, 81, 85, 89, 93, 97\\
Eliminating prome members, have:\\
1, 5, 9, 13, 17, 21, 29, 33, 37, 41, 49, 53, 57, 61, 69, 73, 77, 89, 93, 97

% 5
\subsection{}
What is the prime-power decomposition of 7950?\\~\\
$2*3*5^2*53$

\section{Problems}

% 1
\subsection{}
Find the prime-power decompositions of 1234, 34560, and 111111.\\~\\
2 * 617\\
$2^8*3^3*5$\\
3 * 7 * 11 * 13 * 37

% 2
\subsection{}
Find the prime-power decompositions of 2345, 45670, and 999999999999.\\~\\
5 * 7 * 67\\
2 * 5 * 4567\\
$3^3 * 7 * 11 * 13 * 37 * 101 * 9901$

% 3
\subsection{}
Tartaglia (1556) claimed that the sums
\begin{equation}
    1 + 2 + 4, \quad 1 + 2 + 4 + 8, \quad 1 + 2 + 4 + 8 + 16, \quad ...
\end{equation}
are alternately prime and composite. Show that he was wrong.\\~\\
The sums are of the form $f(n) = \Sigma_{k=0}^n 2^k$.
This holds until $f(7) = 255$, which is composite.
However $f(8) = 511$ is divisible by 7, so it is not prime.\\

% 4
\subsection{}
\textbf{(a)} DeBouvelles (1509) claimed that one or both of $6n + 1$ and $6n - 1$
are primes for all $n \geq 1$. Show that he was wrong.\\
\textbf{(b)} Show that there are infinitely many $n$ such that both $6n - 1$ and
$6n + 1$ are composite.\\~\\
\textbf{(a)} When $n = 24$, $6n + 1 = 145$, which is divisible by 5, and
$6n - 1 = 143$, which is divisble by 11.\\
\textbf{(b)} Already know that this property holds for some $n$, namely $n = 24$.
Suppose $\exists n' \ni n'$ is the greatest $n$ for which the property holds.
Observe that $6(n' + k) - 1 = (6n' - 1) + 6k$ and $6(n' + k) + 1 = (6n' + 1) + 6k$.
We know that both $(6n' - 1)$ and $(6n' + 1)$ are composite,
so can be written as $p_1*p_2*...*p_n$ and $q_1*q_2*...*q_n$.
So, if we pick $k \ni k$ shares a divisor with both $(6n' - 1)$ and $(6n' + 1)$,
we know that the property won't hold for $n' + k$.
Even with no $p_i = q_j \forall i, j$, we can manufacture such a $k$ by picking an
arbitrary product of any combination of $p_i$s and $p_q$s.
So, $n'$ is not the greatest $n$ for which the property holds.

% 5
\subsection{}
Prove that if $n$ is a square, then each exponent in its prime-power
decomposition is even.\\~\\
If $n$ is a square, we know that $\exists k \ni k^2 = n$.
Let $k = p_1^{e_1}*p_2^{e_2}*...*p_k^{e_k}$.
Then $k^2 = p_1^{2e_1}*p_2^{2e_2}*...*p_k^{2e_k}$.
By \textbf{Theorem 2}, this is the unique prime decomposition of n,
and all $e_i$ are even.

% 6
\subsection{}
Prove that if each exponent in the prime-power decomposition of $n$ is even,
then $n$ is a square.\\~\\
We write $n = p_1^{2e_1}*p_2^{2e_2}*...*p_k^{2e_k}$.
This can be rewritten as $p_1^{e_1}*p_1^{e_1}*p_2^{e_2}*p_2^{e_2}*...*p_k^{e_k}*p_k^{e_k} =
(p_1^{e_1}*p_2^{e_2}*...*p_k^{e_k})^2$.
So, $\exists k = p_1^{e_1}*p_2^{e_2}*...*p_k^{e_k} \ni k^2 = n$.
 
% 7
\subsection{}
Find the smallest integer divisble by 2 and 3 which is simultaneously a square
and a fifth power.\\~\\
We can show analogously to the previous exercise that if each exponent in the
prime-power decomposition of $n$ is divisible by $d$, then $\exists k \ni k^d = n$.
The smallest $d'$ for which this holds for both $d_1 = 2$ and $d_2 = 5$ is 10.
So $2^{10} * 3^{10} = 60466176$ is the smallest integer with such a property.

% 8
\subsection{}
If $d|ab$, does it follow that $d|a$ or $d|b$?\\~\\
No, for example if $d = ab$ and $a, b > 1$, then $d|ab$, but $d \nmid a$, $d \nmid b$.

% 9
\subsection{}
Is it possible for a prime $p$ to divide both $n$ and $n + 1 \; (n \geq 1)$?\\~\\
$p|n \Rightarrow \exists k \ni n = pk$.
Then $n + 1 = pk + 1$.
If $p|pk + 1$, then $p|pk$ and $p|1$.
But there is no prime that divides 1.

% 10
\subsection{}
Prove that $n(n + 1)$ is never a square for $n > 0$.\\~\\
$n(n + 1) = n^2 + n$. The number $n^2$ is certainly a square.
Since $n^2 + n > n^2$, if $n^2 + n$ is a square, it must be of some $k > n$.
The smallest such $k \in \Z$ is $n + 1$.
However, $(n + 1)^2 = n^2 + 2n + 1 > n^2 + n$.
The inequality will hold for any other $k > n + 1$ as well, so there is no such $k$.

% 11
\subsection{}
\textbf{(a)} Verify that $2^5*9^2 = 2592$.\\
\textbf{(b)} Is $2^5*a^b = [25ab]$ possible for other $a, b$?
(Here, $[25ab]$ denotes the digits of $2^5*a^b$ and not a product.)\\~\\
\textbf{(a)} Sure.\\
\textbf{(b)} $2^5 = 32$.
Let's examine the range in which $a^b$ must fall to produce a 4 digit $[25ab]$.
The ceiling of $2510 / 32$ is $79$
(never will $a = 0$ have the desired property, since $32 * 0^b = 0$, or at best, and
debatedly so, $32$ when $b = 0$; this hardly makes a difference for what follows).
The floor of $2599 / 32$ is $81$, a number we're familiar with as $9^2$.
So $a^b \in \{79, 80, 81\}$.
Let's write the prime-power decompositions of each:
\begin{align}
    79 &= 79^1 \\
    80 &= 2^4 * 5 \\
    81 &= 3^4
\end{align}
$79$ can be written as $79^1$, but then $[25ab] \geq 25000 > 2599$.\\
$80$ can be written as $80^1$, but then $[25ab] \geq 25000 > 2599$.\\
$81$ can be written as $3^4$ as well as $9^2$, but since $32 * 81 = 2592$,
this alternative representation of $81$ does not have the desired property.
$81^1$, like the previous $b = 1$ cases, will also not fulfill the property.

% 12
\subsection{}
Let $p$ be the least prime factor of $n$, where $n$ is composite.
Prove that if $p > n^{1/3}$, then $n/p$ is prime.\\~\\
$n/p$ can only be prime if $n = p * z$ for some prime $z$.
We have $p > n^{1/3} \Rightarrow p^3 = p * p^2 > p * z$.
So we know that $z < p^2$. Suppose $z$ is composite.
Then, by \textbf{Lemma 3}, it must have a divisor $d \ni 1 < d \leq z^{1/2} < p$.
But, if there were such a $d$, then $p$ would not be the least prime factor of $n$.

% 13
\subsection{}
True or false? If $p$ and $q$ divide $n$, and each is greater than $n^{1/4}$,
then $n/pq$ is prime.\\~\\
\textit{TODO, author gives example to show this is false, but I wish there was a more
elegant way than guessing.}

% 14
\subsection{}
Prove that if $n$ is composite, then $2^n - 1$ is composite.\\~\\
Observe that $2^n - 1 = (2 - 1)(2^{n-1} + 2^{n-2} + ... + 1)$.
If $n$ is composite, than it can be written as $ab$ for some $a, b \in \Z$.
So, $(2^a)^b - 1 = (2^a - 1)(2^{a(b-1)} + 2^{a(b-2)} + ... + 1)$.
This is composite by definition.

% 15
\subsection{}
Is it true that if $2^n - 1$ is composite, then $n$ is composite?\\~\\
$2^{11} - 1 = 2047 = 23 * 89$ is apparently the famous counterexample...
Dunno if there is a good (feasible for me) "analytic" way to show this.


\end{document}
EOF
