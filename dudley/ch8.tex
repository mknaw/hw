
\documentclass{article}
\usepackage{amsmath}
\usepackage{amssymb}
\usepackage[a4paper,bindingoffset=0.2in,%
            left=1in,right=1in,top=1in,bottom=1in,%
            footskip=.25in]{geometry}

\begin{document}

\newcommand{\Z}{\mathbb{Z}}
\newcommand{\s}{\sigma}
\newcommand{\Mod}[1]{\ (\mathrm{mod}\ #1)}

\title{8. Perfect Numbers}
\section{Exercises}

% 1
\subsection{}
Verify that 1184 and 1210 are amicable.
\begin{align*}
    \s(1184) &= \s(2^5)\s(37) = 2394 = 1184 + 1210\\
    \s(1210) &= \s(2)\s(5)\s(11^2) = 2394 = 1184 + 1210
\end{align*}

\section{Problems}

% 1
\subsection{}
Verify that 2620, 2924 and 17296, 18416 are amicable pairs.
\begin{align*}
    \s(2620) &= \s(2^2)\s(5)\s(131) = 5544 = 2620 + 2924\\
    \s(2924) &= \s(2^2)\s(17)\s(43) = 5544 = 2620 + 2924
\end{align*}
\begin{align*}
    \s(17296) &= \s(2^4)\s(23)\s(47) = 35712 = 17296 + 18416\\
    \s(18416) &= \s(2^4)\s(1151) = 35712 = 17296 + 18416
\end{align*}

% 2
\subsection{}
It was long thought that even perfect numbers ended alternately in 6 and 8.
Show that this is wrong by verifying that the prefect numbers corresponding to
the primes $2^{13} - 1$ and $2^{17} - 1$ both end in 6.\\~\\
The perfect numbers corresponding to these primes are
$2^{12}(2^{13} - 1) = 2^{25} - 2^{12}$ and $2^{16}(2^{17} - 1) = 2^{33} - 2^{16}$
\begin{align*}
    2 &\equiv 2 \Mod{10}\\
    2^2 &\equiv 4 \Mod{10}\\
    2^4 &\equiv 6 \Mod{10}\\
    2^8 &\equiv 6 \Mod{10}\\
    2^{16} &\equiv 6 \Mod{10}\\
    2^{32} &\equiv 6 \Mod{10}\\
    2^{25} &\equiv 2 \Mod{10}\\
    2^{12} &\equiv 6 \Mod{10}\\
    2^{25} - 2^{12} &\equiv -4 \equiv 6 \Mod{10}\\
    2^{33} &\equiv 2 \Mod{10}\\
    2^{16} &\equiv 6 \Mod{10}\\
    2^{33} - 2^{16} &\equiv -4 \equiv 6 \Mod{10}
\end{align*}

% 3
\subsection{}
Classify the integers $2, 3, ..., 21$ as abundant, deficient, or perfect.\\~\\
\begin{gather*}
    \s(2) = 3 \Rightarrow deficient, \quad \s(3) = 4 \Rightarrow deficient,\\
    \s(4) = 7 \Rightarrow deficient, \quad \s(5) = 6 \Rightarrow deficient,\\
    \s(6) = 12 \Rightarrow perfect, \quad \s(7) = 8 \Rightarrow deficient,\\
    \s(8) = 15 \Rightarrow deficient, \quad \s(9) = 13 \Rightarrow deficient,\\
    \s(10) = 18 \Rightarrow deficient, \quad \s(11) = 12 \Rightarrow deficient,\\
    \s(12) = 28 \Rightarrow abundant, \quad \s(13) = 14 \Rightarrow deficient,\\
    \s(14) = 24 \Rightarrow deficient, \quad \s(15) = 24 \Rightarrow deficient,\\
    \s(16) = 31 \Rightarrow deficient, \quad \s(17) = 18 \Rightarrow deficient,\\
    \s(18) = 39 \Rightarrow abundant, \quad \s(19) = 20 \Rightarrow deficient,\\
    \s(20) = 42 \Rightarrow abundant, \quad \s(21) = 32 \Rightarrow deficient.
\end{gather*}

% 4
\subsection{}
Classify the integers $402, 403, ..., 421$ as abundant, deficient, or perfect.\\~\\
\textit{TODO sounds boring...}

% 5
\subsection{}
If $\s(n) = kn$, then $n$ is called a \textit{k-perfect number}.
Verify that 672 is 3-perfect and $2,178,540 = 2^2 * 3^2 * 5 * 7^2 * 13 * 19$
is 4-perfect.
\begin{align*}
    \s(672) &= \s(2^5)\s(3)\s(7) = 2016 = 3 * 672\\
    \s(2178540) &= \s(2^2)\s(3^2)\s(5)\s(7^2)\s(13)\s(19) = 8714160 = 4 * 2178540
\end{align*}

% 6
\subsection{}
Show that no number of the form $2^a3^b$ is 3-perfect.\\~\\
In order for this to work, would need $\s(2^a3^b) = 2^a3^{b+1}$.
Note that $\s(2^a3^b) = \s(2^a)\s(3^b)
= \frac{1}{2}(2^{a + 1} - 1)(3^{b + 1} - 1)
= \frac{1}{2}(2^{a + 1}3^{b + 1} - 2^{a + 1} - 3^{b + 1} + 1)
= 2^{a}3^{b + 1} - \frac{1}{2}(2^{a + 1} + 3^{b + 1} - 1)$.
This would only equal $2^a3^{b+1}$ if $2^{a + 1} + 3^{b + 1} = 1$,
but since the term is at minimum $4 + 9 = 13$ with $a, b = 1$,
this cannot be the case.

% 7
\subsection{}
Let us say that $n$ is \textit{superperfect} if and only if $\s(\s(n)) = 2n$.
Show that if $n = 2^k$ and $2^{k + 1} - 1$ is prime, then $n$ is superperfect.\\~\\
If $n = 2^k$, then $\s(n) = \s(2^k) = 2^{k + 1} - 1$.
Since $ 2^{k + 1} - 1$ is prime, we know that $\s(2^{k + 1} - 1) = 2^{k + 1} = 2*2^k$.

% 8
\subsection{}
It was long thought that every abundant number was even.
Show that 945 is abundant, and find another abundant number of the form $3^a*5*7$.\\~\\
$\s(945) = \s(3^3)\s(5)\s(7) = 1200 > 945$.\\
Try $n = 3^4 * 5 * 7 = 2835$. $\s(2835) = \s(3^4)\s(5)\s(7) = 4235 > 2835$.

% 9
\subsection{}
In 1575, it was observed that every even perfect number is a triangular number.
Show that this is so.\\~\\
If $n$ is an even perfect number, then we know by Euler's theorem that
$n = 2^{p - 1}(2^p - 1)$.
We can rewrite this, multiplying by $\frac{2}{2}$, as
$\frac{2^{p}(2^p - 1)}{2}$.
So, setting $k$ as $2^p - 1$, this fits the formula for a triangular number,
namely $n = k(k + 1)/2$.

% 10
\subsection{}
In 1652, it was observed that
\begin{align*}
    6 &= 1 + 2 + 3,\\
    28 &= 1 + 2 + 3 + 4 + 5 + 6 + 7,\\
    496 &= 1 + 2 + 3 + ... + 31.
\end{align*}
Can this go on?\\~\\
A sum of consecutive integers $1 + ... + k = \frac{k(k + 1)}{2}$,
and in the previous exercise we've shown that every perfect even number is
a triangular number, so sure.

% 11
\subsection{}
Let
\begin{align*}
    p &= 3*2^e - 1,\\
    q &= 3*2^{e - 1} - 1,\\
    r &= 3^2*2^{2e - 1} - 1,
\end{align*}
where $e$ is a positive integer.
If $p$, $q$, and $r$ are all prime, show that $2^epq$ and $2^er$ are amicable.\\~\\
Assume all $p$, $q$, $r > 2$, which is the case for $e > 1$.
know that $\s(2^epq) = \s(2^e)\s(p)\s(q)$ and $\s(2^er) = \s(2^e)\s(r)$.
In order for the two numbers to be amicable, need
$\s(r) = 1 + r = 3^2*2^{2e - 1} = (p + 1)(q + 1)$:
\begin{align*}
    (p + 1)(q + 1) &= (3*2^e)(3*2^{e - 1})\\
    &= 3^2*2^{2e - 1}\\
    &= r + 1
\end{align*}
For $e = 1$, have $2^epq = 20$ with $\s(2^epq) = 42$
and $2^er = 34$, with $\s(2^er) = 54$?

% 12
\subsection{}
Show that if $p > 3$ and $2p + 1$ is prime, then $2p(2p + 1)$ is deficient.\\~\\
$\s(2p(2p + 1)) = \s(2)\s(p)\s(2p + 1) = 6 * (p + 1)^2 = 6p^2 + 12p + 6$.
We want to compare this with $2 * 2p(2p + 1) = 8p^2 + 4p$.
For $p = 5$, the smallest acceptable $p$, 216 < 300.
Consider the difference between the two, $2p^2 - 8p - 6$.
The derivative of this with respect to $p$ is $4p - 8$, which is $> 0$ for $p > 3$.
So, $2 * 2p(2p + 1) > \s(2p(2p + 1))$, and $2p(2p + 1)$ is deficient.

% 13
\subsection{}
Show that all even perfect numbers end in 6 or 8.\\~\\
By Euler's theorem, if $n$ is an even perfect number, then it
can be written $n = 2^{p - 1}(2^p - 1)$.
Consider the sequence of powers of $2 \Mod{10}$:
\begin{align*}
    2^1 &\equiv 2 \Mod{10}\\
    2^2 &\equiv 4 \Mod{10}\\
    2^3 &\equiv 8 \Mod{10}\\
    2^4 &\equiv 6 \Mod{10}\\
    2^5 &\equiv 2 \Mod{10}\\
    2^6 &\equiv 4 \Mod{10}\\
    2^7 &\equiv 8 \Mod{10}\\
    2^8 &\equiv 6 \Mod{10}\\
    ...
\end{align*}
Subsequent powers of 2 will follow this cycle of 2, 4, 8, 6 $\Mod{10}$.
$2^{p - 1}$ will have an even exponent when $p > 2$,
so it will be will be $\equiv$ to either 4 or 6 $\Mod{10}$.
If it's $\equiv 4$, then $2^p - 1 \equiv 7$ and
$2^{p - 1}(2^p - 1) \equiv 4 * 7 \equiv 8 \Mod{10}$.
If it's $\equiv 6$, then $2^p - 1 \equiv 1$ and
$2^{p - 1}(2^p - 1) \equiv 1 * 6 \equiv 6 \Mod{10}$.
For $p = 2$, $2^{p - 1}(2^p - 1) = 6$, which ends with a 6.

% 14
\subsection{}
If $n$ is an even perfect number and $n > 6$, show that the sum of its digits
is congruent to $1 \Mod{9}$.\\~\\
$n$ can be written $d_k10^k + d_{k-1}10^{k-1} + ... + d_110 + d_0$.
Note that $10^k \equiv 1 \Mod{9}$.
So $n \equiv d_k + d_{k-1} + ... + d_1 + d_0 \Mod{9}$.
So if the sum of digits $\Sigma d_i \equiv 1 \Mod{9}$, then $n \equiv 1 \Mod{9}$.
If $n$ is an even perfect number, can write $n = 2^{p - 1}(2^p - 1)$.
Consider the sequence of powers of $2 \Mod{9}$:
\begin{align*}
    2^1 &\equiv 2 \Mod{9}\\
    2^2 &\equiv 4 \Mod{9}\\
    2^3 &\equiv 8 \Mod{9}\\
    2^4 &\equiv 7 \Mod{9}\\
    2^5 &\equiv 5 \Mod{9}\\
    2^6 &\equiv 1 \Mod{9}\\
    2^7 &\equiv 2 \Mod{9}\\
    2^8 &\equiv 4 \Mod{9}\\
    2^8 &\equiv 8 \Mod{9}\\
    2^4 &\equiv 7 \Mod{9}\\
    2^5 &\equiv 5 \Mod{9}\\
    ...
\end{align*}
By a similar argument as the previous exercise,
$2^{p - 1} \equiv 4, 7$, or $1$ and $2^p - 1 \equiv 7, 4$, or $1$, respectively.
Then $4 * 7 \equiv 7 * 4 \equiv 1 \Mod{9}$, and $1 * 1 \equiv 1 \Mod{9}$.

% 15
\subsection{}
If $p$ is odd, show that $2^{p - 1}(2^p - 1) \equiv 1 + 9p(p - 1)/2 \Mod{81}$.\\~\\
\textit{TODO}

\end{document}
EOF
