
\documentclass{article}
\usepackage{amsmath}
\usepackage{amssymb}
\usepackage[a4paper,bindingoffset=0.2in,%
            left=1in,right=1in,top=1in,bottom=1in,%
            footskip=.25in]{geometry}

\begin{document}

\newcommand{\Z}{\mathbb{Z}}
\newcommand{\s}{\sigma}
\newcommand{\Mod}[1]{\ (\mathrm{mod}\ #1)}

\title{6. Fermat's and Wilson's Theorems}
\section{Exercises}

% 1
\subsection{}
Verify that the table is correct as far as it goes, and complete it.
\begin{align*}
    d(11) &= 2\\
    d(12) &= 6\\
    d(13) &= 2\\
    d(14) &= 4\\
    d(15) &= 4\\
    d(16) &= 5
\end{align*}

% 2
\subsection{}
What is $d(p^3)$ Generalize to $d(p^n), n = 4, 5, ...$.\\~\\
$p^3 = p * p * p$ must be divisible by 1, $p$, $p^2$ and $p^3$, so 4.
$d(p^n) = n + 1$.

% 3
\subsection{}
What is $d(p^3q)$? What is $d(p^nq)$ for any positive $n$?\\~\\
$p^3q$ is divisble by 1, $p$, $p^2$, $p^3$, $q$, $pq$, $p^2q$, $p^3q$, so $d(p^3q) = 7$.
For $p^nq$, we'll have divisors of 1, $q$, $n$ terms of $p^k$, and $n$ terms of
$p^kq$, so $d(p^nq) = 2(n + 1)$.

% 4
\subsection{}
Calculate $d(240)$.\\~\\
$d(240) = d(2^4)*d(3)*d(5) = 5 * 2 * 2 = 20$.

% 5
\subsection{}
Verify that the following table is correct as far as it goes, and complete it.
\begin{align*}
    \s(9) &= 13\\
    \s(10) &= 18\\
    \s(11) &= 12\\
    \s(12) &= 28\\
    \s(13) &= 14\\
    \s(14) &= 24\\
\end{align*}

% 6
\subsection{}
What is $\s(p^3)$? $\s(pq)$, where $p$ and $q$ are different primes?\\~\\
$\s(p^3) = 1 + p + p^2 + p^3$. $\s(pq) = 1 + p + q + pq$.

% 7
\subsection{}
Show that $\s(2^n) = 2^{n + 1} - 1$.\\~\\
$\s(2^n) = 1 + 2 + ... + 2^n$, so
$2\s(2^n) - \s(2^n) = \s(2^n) = 2^{n + 1} - 1$.

% 8
\subsection{}
What is $\s(p^n)$, $n = 1, 2, ...$?\\~\\
$\s(p^n) = 1 + p + ... + p^n$, or, using a well-known alternate defintion for
a geometric series, $\s(p^n) = \frac{p^{n + 1} - 1}{p - 1}$.

% 9
\subsection{}
Calculate $\s(240)$.\\~\\
$\s(240) = \s(2^4)*\s(3)*\s(5) = (1 + 2 + 4 + 8 + 16)(1 + 3)(1 + 5) = 744$.

% 10
\subsection{}
Compute $f(n)$ for $n = 13, 14, ..., 24$.
\begin{gather*}
    f(13) = 1, \quad f(14) = 1\\
    f(15) = 1, \quad f(16) = 32\\
    f(17) = 1, \quad f(18) = 6\\
    f(19) = 1, \quad f(20) = 4\\
    f(21) = 1, \quad f(22) = 1\\
    f(23) = 1, \quad f(24) = 12
\end{gather*}

\section{Problems}

% 1
\subsection{}
Calculate $d(42)$, $\s(42)$, $d(420)$, and $\s(420)$.\\~\\
$d(42) = d(2)*d(3)*d(7) = 8$.\\
$\s(42) = \s(2)*\s(3)*\s(7) = 3 * 4 * 8 = 96$.\\
$d(420) = d(2^2)*d(3)*d(5)*d(7) = 24$.\\
$\s(420) = \s(2^2)*\s(3)*\s(5)*\s(7) = 7 * 4 * 6 * 8 = 1344$.

% 2
\subsection{}
Calculate $d(540)$, $\s(540)$, $d(5400)$, and $\s(540)$.\\~\\
$d(540) = d(2^2)*d(3^3)*d(5) = 3 * 4 * 2 = 24$.\\
$\s(540) = \s(2^2)*\s(3^3)*\s(5) = 7 * 34 * 6 = 1428$.\\
$d(5400) = d(2^3)*d(3^3)*d(5^2) = 4 * 4 * 3 = 48$.\\
$\s(5400) = \s(2^3)*\s(3^3)*\s(5^2) = 15 * 34 * 31 = 15810$.\\

% 3
\subsection{}
Calculate $d$ and $\s$ of $10115 = 5 * 7 * 17^2$ and $100115 = 5 * 20023$.\\~\\
$d(10115) = d(5)*d(7)*d(17^2) = 2 * 2 * 3 = 12$.\\
$\s(10115) = \s(5)*\s(7)*\s(17^2) = 6 * 8 * 307 = 14736$.\\
$d(100115) = d(5)*d(20023) = 2 * 2 = 4$.\\
$\s(100115) = \s(5)*\s(20023) = 6 * 20024 = 120144$.

% 4
\subsection{}
Calculate $d$ and $\s$ of $10116 = 2^2 * 3^2 * 281$ and $100116 = 2^2 * 3^5 * 103$.\\~\\
$d(10116) = d(2^2)*d(3^2)*d(281) = 3 * 3 * 2 = 18$.\\
$\s(10116) = \s(2^2)*\s(3^2)*\s(281) = 7 * 13 * 282 = 25662$.\\
$d(100116) = d(2^2)*d(3^5)*d(103) = 3 * 6 * 2 = 36$.\\
$\s(100116) = \s(2^2)*\s(3^5)*\s(103) = 7 * 286 * 104 = 208208$.\\

% 5
\subsection{}
Show that $\s(n)$ is odd if $n$ is a power of two.\\~\\
Can write $\s(n) = \s(k^2) = 1 + k + k^2$.
If $k$ is even, then $k^2$ is even, and $1 + k + k^2$ is odd.
If $k$ is odd, then $k^2$ is odd, $k + k^2$ is even, and $1 + k + k^2$ is odd.

% 6
\subsection{}
Prove that if $f(n)$ is multiplicative, then so is $f(n)/n$.\\~\\
If $f(n)$ is multiplicative, then $f(n) = f(p_1^{e_1})f(p_1^{e_1})...f(p_1^{e_1})$
with $n = p_1^{e_1}*p_2^{e_2}*...*p_k^{e_k}$.
So, $f(n)/n =
\frac{f(p_1^{e_1})f(p_2^{e_2})...f(p_k^{e_k})}{p_1^{e_1}*p_2^{e_2}*...*p_k^{e_k}}
= \frac{f(p_1^{e_1})}{p_1^{e_1}}*\frac{f(p_2^{e_2})}{p_2^{e_2}}
*...*\frac{f(p_k^{e_k})}{p_k^{e_k}}$.
This implies $f(n)/n$ is multiplicative.

% 7
\subsection{}
What is the smallest integer $n$ such that $d(n) = 8$? Such that $d(n) = 10$?\\~\\
Consider that $8 = 2^3$, so we could have the product of any 3 primes,
for example $d(2)*d(3)*d(5)$. We also can't have any divisor that is a
square prime, because $d(p^2) = 3$ and $3 \nmid 8$.
But, we can have a cubic divisor, since $d(p^3) = 4$,
and as it turns out, $2 * 2^2 < 2 * 5$, so $24 = 2^3 * 3$ is the smallest such $n$.\\
For $d(n) = 10$, note that $10 = 2 * 5$.
To $d(n) = 10$, have to have a $d(k^4)$ somewhere, and it might as well be $2^4$.
So, the smallest such $n$ is $48 = 2^4 * 3$.

% 8
\subsection{}
Does $d(n) = k$ have a solution $n$ for each $k$?\\~\\
Any $k$ would have at least the solution $n = p^{k - 1}$ for any prime $p$.

% 9
\subsection{}
In 1644, Mersenne asked for a number with 60 divisors.
Find one smaller than 10,000.\\~\\
Consider that $60 = 2^2 * 3 * 5$.
So in our number we need at one divisor that is a prime raised to the fourth power,
and one that is a square.
So, it would be best to allocate these to the lowest primes, 2 and 3 respectively.
We could take care of the $2^2$ term with a prime raised to the third power or
two distinct prime powers.
Since 5 is the next power and $5^3 > 5 * 7$, we opt for the latter.
$5040 = 2^4*3^2*5*7$ is such a number.

% 10
\subsection{}
Find infinitely many $n$ such that $d(n) = 60$.\\~\\
For any $n = p^{59}$, $d(n) = 60$.
Since there are infinitely many primes $p$, there are infinitely many $n = p^{59}$.
Or you could have infinitely many permutations of the forms $p_1^3*p_2^2*p_3^4$
or $p_1*p_2*p_3^2*p_4^4$.

% 11
\subsection{}
If $p$ is an odd prime, for which $k$ is $1 + p + ... + p^k$ odd?\\~\\
If $p$ is odd, then any $p^i$ is odd as well.
Also, any sums pairs like $p^i + p^j$ are even.
Since we have the $1$ term, we want $p + ... + p^k$ to be even.
So, we would want to be able to group it into pairs.
This is possible if $k$ is even.

% 12
\subsection{}
For which $n$ is $\s(n)$ odd?\\~\\
If $n = p^k$ and $k$ even, we have odd $\s(n)$, like in the previous exercise.
The same could be said for any $k$ if $p = 2$.
If $n$ cannot be written as a power of a single prime $p$,
it can still be written as $n = p_1^{e_1}*...*p_k^{e_k}$,
and $\s(n) = \s(p_1^{e_1})**...*\s(p_k^{e_k})$.
In order for that product to be odd, each term would have to be odd, i.e.
each term $p_i^{e_i}$ in the prime-power decomposition of $n$ would either
require $p_i = 2$ or $2|e_i$.

% 13
\subsection{}
If $n$ is a square, show that $d(n)$ is odd.\\~\\
Write $n = k^2$, then $d(n) = d(k^2) = d((p_1^{e_1}*...*p_k^{e_k})^2) = 
d(p_1^{2e_1}*...*p_k^{2e_k}) = d(p_1^{2e_1})*...*d(p_k^{2e_k}) =
(1 + 2e_1)*...*(1 + 2e_k)$.
Since each term of this product is odd, the whole product is odd.

% 14
\subsection{}
If $d(n)$ is odd, show that $n$ is a square.\\~\\
$d(n) = d(p_1^{e_1}*...*p_k^{e_k}) = d(p_1^{e_1})*...*d(p_k^{e_k}) = 
(1 + e_1)*...*(1 + e_k)$.
Since we know $d(n)$ is odd, we know that none of the terms of the product can be
even, and $e_i$ must be even.
But, if $e_i$ is even, then it can be written as $2e_i'$,
and $n$ can be written $p_1^{2e_1'}*...*p_k^{2e_k'} =
(p_1^{e_1'}*...*p_k^{e_k'})^2$.

% 15
\subsection{}
Observe that $1 + 1/3 = 4/3$; $1 + 1/2 + 1/4 = 7/4$;
$1 + 1/5 = 6/5$; $1 + 1/2 + 1/3 + 1/6 = 15/6$; $1 + 1/7 = 8/7$;
and $1 + 1/2 + 1/4 + 1/8 = 15/8$.
Guess and prove a theorem.\\~\\
$\Sigma_{d|n}1/d = \s(n)/n$.\\
Consider the sum over the divisors of $n$: $1/1 + 1/d_1 + ... + 1/d_k + 1/n$.
To perform the addition, we want to have all fractions have the same denominator.
$n$ is the natural candidate for this, as by definition
$\exists d_i' \ni d_id_i' = n$.
Furthermore, the set of these $d_i'$s (call it $D'$) is equivalent to the
set of $d_i$s ($D$):
If there were a $d_i'$ that is not in $D$, it would contradict the construction
of $D$ as the set of all divisors of $n$.
For $d_i$s, we still know that $d_i' = n/d_i \in D$,
and thus $\exists d_j' \in D' \ni \frac{n}{d_i}d_j' = n$, namely $d_j' = d_i$.
So all $d_i$s must be in $D'$ too.
So, we now have $n/n + d_1'/n + ... + d_k'/n + 1/n = 
\frac{1}{n}(1 + d_k' + ... + d_1' + n) = \s(n)/n$.

% 16
\subsection{}
Find infinitely many $n$ such that $\s(n) \leq \s(n - 1)$.\\~\\
Consider $n$ that are odd primes.
By the definition of prime numbers, $\s(n) = 1 + n$.
Since $n$ is odd, then we know that $n - 1$ must be even,
thus it must have have 2 among its divisors.
So, $\s(n - 1) = 1 + 2 + (n - 1) + r$, where $r$ represents the remaining terms
in the summation and $r > 0$.
$1 + 2 + (n - 1) + r = 2 + n + r \geq 1 + n$.
Since there are infinitely many odd primes, there are infinitely many such $n$.

% 17
\subsection{}
If $N$ is odd, how many solutions does $x^2 - y^2 = N$ have?\\~\\
Write $x^2 - y^2 = (x + y)(x - y) = ab = N$.
The number of divisors of $N$ is $d(N)$, so there are $d(N)$ possible positive
values for $a$ (with appropriate $b$s) and $d(N)$ negative ones
(with appropriate negative $b$s).
The solutions to $x + y = a$, $x - y = b$ are $x = \frac{a + b}{2}$,
$y = \frac{a - b}{2}$.
So, should have $2d(N)$ such $(x, y)$.
The comments say something about $a, b$ needing to be odd to ensure
unique $(x, y)$ pairs, but I'm not really seeing it!

% 18
\subsection{}
Develop a formula for $\s_2(n)$, the sum of the squares of the positive divisors
of $n$.\\~\\
Observe that for prime $p$, $\s_2(p^k) = 1 + p^2 + p^4 + ... + p^{2k}
= 1 + (p^2)^1 + (p^2)^2 + ... + (p^2)^k = \frac{(p^2)^{k+1} - 1}{p^2 - 1}$.
For primes $p$, $q$, $\s_2(p^kq^j) =
1 + p^2 + ... + p^{2k} + p^2q^2 + ... + p^2q^{2j}
+ ... + p^{2k}q^2 + ... + p^{2k}q^{2j} + q^2 + ... + q^{2j}
= (1 + p^2 + ... + p^{2k})(1 + q^2 + ... + q^{2j})
= \frac{(p^2)^{k+1} - 1}{p^2 - 1}\frac{(q^2)^{q+1} - 1}{q^2 - 1}$.
So, since any number $n$ can be represented as a prime-power decomposition
$p_1^{e_1}...p_k^{e_k}$, $\s_2(n) =
\frac{(p_1^2)^{e_1+1} - 1}{p_1^2 - 1}... \frac{(p_k^2)^{e_k+1} - 1}{p_k^2 - 1}$.

% 19
\subsection{}
Guess a formula for
\begin{equation*}
    \s_k(n) = \Sigma_{d|n}d^k,
\end{equation*}
where $k$ is a positive integer.\\~\\
By the same logic as before, $\s_k(n) =
\frac{(p_1^k)^{e_1+1} - 1}{p_1^k - 1}... \frac{(p_k^k)^{e_k+1} - 1}{p_k^k - 1}$.


% 20
\subsection{}
Show that the product of the positive divisors of $n$ is $n^{d(n)/2}$.\\~\\
Consider the easy case where $n = p^k$ and denote our product function as $f$.
We have $f(p^k) = p*p^2*...*p^k = p^{1 + 2 + ... + k}$.
We are trying to show, for this special case, that
$p^{1 + 2 + ... + k} = p^{\frac{k}{2}d(p^k)}$,
or equivalently, $1 + 2 + ... + k = \frac{k}{2}d(p^k)$.
The sum of consecutive integers on the left side can be written as
$\frac{k(k + 1)}{2}$.
We also know that $d(p^k) = k + 1$, making the right side $\frac{k(k + 1)}{2}$,
so we've shown that the relation holds for the special case of $n = p^k$.
Now consider $f(p^kq^j) = p...p^k*pq...pq^j*p^kq...p^kq^j*q...q^j$.
In this product we have 1 term of $p...p^k$ with no $q$ terms,
and $j$ terms of $p...p^k$ with terms $\in \{q, ..., q^j\}$.
So we have $j + 1$ terms of $p...p^k$ and, symmetrically,
$k + 1$ terms of $q...q^j$.
Knowing that $p...p^k = f(p^k) = (p^k)^{d(p^k)/2}$ and that
$k + 1 = d(p^k), j + 1 = d(q^j)$, can write
$f(p^kq^j) = (p^k)^{\frac{1}{2}d(q^j)d(p^k)}(q^j)^{\frac{1}{2}d(q^j)d(p^k)}
= (p^kq^j)^{\frac{1}{2}d(q^j)d(p^k)} = (p^kq^j)^{\frac{1}{2}d(p^kq^j)}$
Assume that $f$ is multiplicative up to $r$ for composite numbers
that can be written $p_1^{e_1}...p_k^{e_r}$.
Observe that for $f(p_1^{e_1}...p_k^{e_r}p_{r+1}^{e_{r+1}})$,
we can apply similar logic as we did for $p^kq^j$,
namely that there will be $e_{r+1} + 1 = d(p_{r+1}^{e_{r+1}})$
terms of $f(p_1^{e_1}...p_k^{e_r})$ and $d(p_1^{e_1}...p_k^{e_r})$
terms of $f(p_{r+1}^{e_{r+1}})$, and we can extract the common exponent
$d(p_1^{e_1}...p_k^{e_r}p_{r+1}^{e_{r+1}})/2$.
Since any number can be written as such a prime-power decomposition,
$f(n) = n^{d(n)/2}$.


\end{document}
EOF
