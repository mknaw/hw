\documentclass{article} \usepackage{amsmath}
\usepackage{amssymb}
\usepackage[a4paper,bindingoffset=0.2in,%
            left=1in,right=1in,top=1in,bottom=1in,%
            footskip=.25in]{geometry}

\begin{document}

\newcommand{\Z}{\mathbb{Z}}
\newcommand{\s}{\sigma}
\newcommand{\p}{\phi}
\newcommand{\e}{\equiv}
\newcommand{\m}[1]{\ (\mathrm{mod}\ #1)}

\title{13. Numbers in Other Bases}
\section{Exercises}

% 1
\subsection{}
Write 31 and 33 as sums of distinct powers of two.
\begin{align*}
    31 &= 2^4 + 2^3 + 2^2 + 2^1 + 2^0\\
    33 &= 2^5 + 2^0
\end{align*}

% 2
\subsection{}
What is $r$ if $n = 74$? If $n = 174$?\\~\\
74 is between $2^6$ and $2^{6+1}$, i.e. $r = 6$.
164 is between $2^7$ and $2^{7+1}$, i.e. $r = 7$.

% 3
\subsection{}
Show that $r$ is different from any of $e_1, e_2, ..., e_k$.\\~\\
If there were an $e_i = r$, then
$n - 2^r = 2^{e_1} + 2^{e_2} + ... + 2^{e_k} \geq 2^r$.
But, if that were the case, then $n \geq 2^{r+1}$, which contradicts
our choice of $r$.

% 4
\subsection{}
Evaluate $1001_2$, $111_2$, and $1000000_2$.
\begin{align*}
    1001_2 &= 2^0 + 2^3 = 9\\
    111_2 &= 2^0 + 2^1 + 2^2 = 7\\
    1000000_2 &= 2^6 = 64
\end{align*}

% 5
\subsection{}
Write 2, 20, and 200 in the base 2.
\begin{align*}
    2 &= 10_2\\
    20 &= 10100_2\\
    200 &= 111000_2
\end{align*}

% 6
\subsection{}
Complete the proof.\\~\\
Since $-1 \leq d_0 - e_0 \leq 1$, the only possible $d_0 - e_0$
that is divisible by $b$ is 0.
So, can drop this $d_0 - e_0$ term and divide the remaining terms
by $b$. Rinse and repeat.

\section{Problems}

% 1
\subsection{}
Write 1492 in base 2, 3, 7, 9, 11.\\~\\
$1492 = 10111010100_2 = 2001021_3 = 4231_7 = 2037_9 = 1137_{11}$.

% 2
\subsection{}
Write 1776 in base 4, 5, 6, 8, 11.\\~\\
$1776 = 123300_2 = 24101_5 = 12120_6 = 3360_8 = 1375_{11}$.

% 3
\subsection{}
Write in base 10:
\begin{gather*}
    3141_5, \quad 3141_7, \quad 3141_{11}, \quad 3141_{12}
\end{gather*}
\begin{align*}
    3141_5 &= 421\\
    3141_7 &= 1107\\
    3141_{11} &= 4159\\
    3141_{12} &= 5377
\end{align*}

% 4
\subsection{}
Write in base 10:
\begin{gather*}
    1215_6, \quad 1215_8, \quad 1215_9, \quad 1215_{20}
\end{gather*}
\begin{align*}
    1215_6 &= 299\\
    1215_8 &= 653\\
    1215_9 &= 905\\
    1215_{20} &= 8825
\end{align*}

% 5
\subsection{}
Solve for $x$:
\begin{gather*}
    123_4 = x_5, \quad 234_5 = x_6, \quad 123_x = 1002_4
\end{gather*}
\begin{align*}
    123_4 &= 27 = 102_5\\
    234_5 &= 69 = 153_6\\
    1002_4 &= 66 = 123_7
\end{align*}

% 6
\subsection{}
Solve for $x$:
\begin{gather*}
    345_6 = x_7, \quad 456_x = 2201_7, \quad 2x3_4 = 1x10_3
\end{gather*}
\begin{gather*}
    345_6 = 137 = 254_7\\
    4x^2 + 5x - 779 = 0\\
    2x3_4 = 35 + 4x = 30 + 9x \Rightarrow x = 1
\end{gather*}
Second one has no solutions $x \in \Z$.

% 7
\subsection{}
Construct a multiplication table in base 7.
\begin{center}
\begin{tabular}{ c|c c c c c c }
    & 1 & 2 & 3 & 4 & 5 & 6 \\ 
    \hline
    1 & 1 & 2  & 3  & 4  & 5  & 6 \\ 
    2 & 2 & 4  & 6  & 11 & 13 & 15 \\ 
    3 & 3 & 6  & 12 & 15 & 21 & 24 \\ 
    4 & 4 & 11 & 15 & 22 & 26 & 33 \\ 
    5 & 5 & 13 & 21 & 26 & 34 & 42 \\ 
    6 & 6 & 15 & 24 & 33 & 42 & 51 
\end{tabular}
\end{center}

% 8
\subsection{}
Construct a multiplication table in base 8.
\begin{center}
\begin{tabular}{ c|c c c c c c c }
    & 1 & 2 & 3 & 4 & 5 & 6 & 7 \\ 
    \hline
    1 & 1 & 2  & 3  & 4  & 5  & 6  & 7 \\ 
    2 & 2 & 4  & 6  & 10 & 12 & 14 & 16 \\ 
    3 & 3 & 6  & 11 & 14 & 17 & 22 & 25 \\ 
    4 & 4 & 10 & 14 & 20 & 24 & 30 & 34 \\ 
    5 & 5 & 12 & 17 & 24 & 31 & 36 & 43 \\ 
    6 & 6 & 14 & 22 & 30 & 36 & 44 & 52 \\
    7 & 7 & 16 & 25 & 34 & 43 & 52 & 61
\end{tabular}
\end{center}

% 9
\subsection{}
All numbers in this problem are in base 9.
Calculate, in base 9 (that is, no conversions to any other base):
\begin{gather*}
    15 + 24 + 33, \quad 1620 - 1453, \quad
    42 \cdot 12, \quad 314 \cdot 152
\end{gather*}
\begin{align*}
    15 + 24 + 33 &= 73\\
    1620 - 1453 &= 156\\
    42 \cdot 12 &= 24 + 480 = 514\\
    314 \cdot 152 &= 728 + 1520 + 46600 = 48848
\end{align*}

% 10
\subsection{}
With the same instructions as in Problem 9, calculate
\begin{gather*}
    16 + 35 + 44, \quad 1453 - 1066, \quad
    53 \cdot 23, \quad 425 \cdot 263
\end{gather*}
\begin{align*}
    16 + 35 + 44 &= 106\\
    1453 - 1066 &= 376\\
    53 \cdot 23 &= 70 + 1260 = 1340\\
    425 \cdot 263 &= 1446 + 5360 + 117300 = 125216
\end{align*}

% 11
\subsection{}
Let $(.d_1d_2d_3...)_b$ stand for $d_1/b + d_2/b^2 + ...$.
Evaluate as rational numbers in base 10:
\begin{gather*}
    (.25)_7, \quad (.333...)_7, \quad (.5454...)_7
\end{gather*}
\begin{align*}
    (.25)_7 &= \frac{2}{7} + \frac{5}{7^2} = \frac{19}{49}\\
    (.333...)_7 &= \frac{3}{7}\left(1 + \frac{1}{7} + \frac{1}{7^2} + ...\right)
    = \frac{3}{7}\cdot\frac{1}{1 - 1/7} = \frac{1}{2}\\
    (.5454...)_7 &= \frac{5}{7} + \frac{4}{7^2} + \frac{5}{7^3} + ...
    = \frac{5}{7}\cdot\frac{1}{1 - 1/7} - \frac{1}{7^2}\cdot\frac{1}{1 - 1/7^2} =
    \frac{5}{6} - \frac{1}{48} = \frac{13}{16}
\end{align*}

% 12
\subsection{}
With the same instructions as in Problem 11, evaluate:
\begin{gather*}
    (.36)_7, \quad (.444...)_7, \quad (.6565...)_7
\end{gather*}
\textit{TODO}

% 13
\subsection{}
In which bases $b$ is $1111_b$ divisible by 5?\\~\\
$1111_b = 1 + b + b^2 + b^3$.
For this to be divisible by 5, $b + b^2 + b^3 \e 4 \m{5}$.
For $b \in (1, 5)$ (clearly $5 \nmid b$, and $b \geq 2$):
\begin{align*}
    b \e 1 \m{5} &\Rightarrow 1 + 1^2 + 1^3 \e 3 \m{5}\\
    b \e 2 \m{5} &\Rightarrow 2 + 2^2 + 2^3 \e 4 \m{5}\\
    b \e 3 \m{5} &\Rightarrow 3 + 3^2 + 3^3 \e 4 \m{5}\\
    b \e 4 \m{5} &\Rightarrow 4 + 4^2 + 4^3 \e 4 \m{5}
\end{align*}
So, any $b \e$ 2, 3, or 4 $\m{5}$ will do.

% 14
\subsection{}
(a) Show that $123_7$, $132_7$, $312_7$, $231_7$, and $213_7$ are even integers.\\
(b) Show that in the base 7, an integer is even if and only if the sum of its
digits is even.\\
(c) In which other bases is it true that if an integer is even, then any
permutation of its digits is even?\\~\\
(a) $123_7 = 49 + 2*7 + 3$, of which the middle term is even and the first and
last are odd, so the sum is even.
$132_7 = 49 + 3*7 + 2$, of which the last term is even and the first two
are odd, so the sum is even.
$312_7 = 3*49 + 7 + 2$, of which the last term is even and the first two
are odd, so the sum is even.
$231_7 = 2*49 + 3*7 + 1$, of which the first term is even and the last two
are odd, so the sum is even.
$213_7 = 2*49 + 7 + 3$, of which the first term is even and the last two
are odd, so the sum is even.\\
(b) Note that $(d_k...d_0)_7 = d_k7^k + ... + d_17 + d_0$.
Know that $7_i \e 1 \m{2}$ for any $i$.
So, $(d_k...d_0)_7 \e d_k + ... + d_1 + d_0 \m{2}$,
or in other words, the number is only even if, and only if,
the sum of its digits is even.\\
(c) We are looking for numbers such that
$(d_k...d_0)_b = d_kb^k + ... + d_1b + d_0 \e d_k + ... + d_0 \m{2}$.
It would be good for this if all $b^k \e 1 \m{2}$.
This is true of odd $b$.

% 15
\subsection{}
An eccentric philanthropist undertakes to give away \$100,000.
He is eccentric because he insists that each of his gifts be a number of dollars
that is a power of two, and he will give no more than one gift of any amount.
How does he distribute the money?\\~\\
We can write 100,000 in base 2 as $11000011010100000_2$.
So, we have 1s for terms corresponding to
$2^5$, $2^7$, $2^9$, $2^{10}$, $2^{15}$, $2^{16}$.
This is how he distributed the money.

% 16
\subsection{}
Prove that every positive integer can be written unique in the form
\begin{equation*}
    n = e_0 + 3e_1 + 3^2e_2 + ... + 3^ke_k
\end{equation*}
for some $k$, where $e_i = -1, 0,$ or 1, $i = 0, 1, ..., k$.\\~\\
Know that we can write any integer as $n = d_0 + 3d_1 + 3^2d_2 + ... + 3^jd_j$
with $d_i \in \{0, 1, 2\}$.
If $d_i = 2$, then $3^i \cdot 2 = 3 \cdot 3^i - 3^i = 3^{i+1} - 3^i$.
So, if we replace such $d_i$ with -1 for the associated $e_i$,
it suffices to increment the $e_{i + 1}$ term by one.
If in turn this term would now exceed 1, we can continue "carrying"
that 1 until we get to a term that does not exceed 1 after the addition.

% 17
\subsection{}
Hexadecimal notation (base 16) uses the digits A, B, C, D, E, and F
for decimals 10, 11, 12, 13, 14, and 15.\\
(a) Convert 3073, 53456, 49370, and 45278 to hexadecimal.\\
(b) Convert CAB, B0B0, DEAF, and A1DE to decimal.\\
(c) Is there a longer hexadecimal word than DEFACADED?\\~\\
(a) C01, D0D0, C0DA, B0DE.\\
(b) 3243, 45232, 57007, 41438.\\
(c) Hard to say... probably not.

% 18
\subsection{}
To convert from decimal to binary, it is convenient to convert first to
octal, and then replace each octal digit with its binary representation.
For example, $1929_{10} = 3611_8 = 11,110,001,001_2$ and
$10,111,010,100_2 = 2724_8 = 1492_{10}$.
Show that this process works in general.\\~\\
Let $(o_j...o_0)_8$ be the octal representation of $n$,
and let $(d_{o_i2}d_{o_i1}d_{o_i0})_2$ be the binary representation of
octal digit $o_i$.
So, $o_i8^i = o_i2^{3i} = (d_2d_1d_0)_22^{3i} = (d_22^2 + d_12 + d_0)2^{3i}
= d_22^{3i + 2} + d_12^{3i + 1} + d_02^{3i}$.
So, any $(o_k...o_0)_8$ can be written
$d_{o_k2}2^{3k + 2} + d_{o_k1}2^{3k + 1} + d_{o_k0}2^{3k} + ...
+ d_{o_02}2^{2} + d_{o_01}2^{1} + d_{o_00} = 
(d_{o_k2}d_{o_k1}d_{o_k0}...d_{o_02}d_{o_01}d_{o_00})_2$

% 19
\subsection{}
Another method of representing integers is in the factorial notation:
\begin{equation*}
    (d_kd_{k-1}...d_1) = d_1 \cdot 1! + d_2 \cdot 2! + ... + d_k \cdot k!, \quad
    0 \leq d_i \leq i
\end{equation*}
(a) Write $(22110)_!$ and $(242120)_!$ in base 10.\\
(b) Write 920 and 1848 in factorial notation.\\
(c) Prove that every positive integer has a unique representation in factorial
notation.\\~\\
(a)
\begin{gather*}
    (22110)_! = 0 \cdot 1! + 1 \cdot 2! + 1 \cdot 3! + 2 \cdot 4! + 2 \cdot 5! = 296\\
    (242120)_! = 0 \cdot 1! + 2 \cdot 2! + 1 \cdot 3! + 2 \cdot 4! + 4 \cdot 5!
    + 2 \cdot 6! = 1978
\end{gather*}
(b) $920 = (113110)_! \quad 1848 = (232000)_!$\\
(c) Let $n$ be the largest integer such that $n! \leq x$, with $x$ the integer that
we want to represent in factorial notation.
We know that the highest digit associated with $n$ can be $n$, because
$(n + 1)n! = (n + 1)!$, which by choice of $n$ is $> x$.
We choose as the factor of $n!$ the highest digit $d_n$ such that $d_nn! \leq x$.
Let the remainder $r = x - d_nn!$, with $r < n!$.
If $r > 0$, there must exist some single highest $n' < n$ such that
$n'! \leq r < (n' + 1)!$; we can iterate this algorithm until $r = 0$.
If at any step we chose a $d_i$ that is not the single greatest possible
$d_i \ni d_ii! \leq x_i$, then the remainder $r$ would be $\geq i!$.
Consider the sequence $1 \cdot 1! + 2 \cdot 2! + ...$.
Know that $1 \cdot 1! < 2!$.
Assume this relationship holds up to $k - 1$.
Then $1 \cdot 1! + ... + (k - 1) \cdot (k - 1)! < k!$.
Now add $k \cdot k!$ to both sides to get
$1 \cdot 1! + ... + k \cdot k! < k! + k \cdot k! = (k + 1)(k + 1)!$.
So, if at any point we did not select the single highest $d_i$ as described in the
algorithm, even if we set all the highest possible $d_i$, namely $d_i = i$,
we still would produce a number that is $< x$.
This means the factorial representation produced by the algorithm is the unique one.

% 20
\subsection{}
Find a base $b$ in which $45_b$ and $55_b$ are squares of consecutive integers.\\~\\
$b = 19$ is such a base.


\end{document}
EOF
