
\documentclass{article}
\usepackage{amsmath}
\usepackage{amssymb}
\usepackage[a4paper,bindingoffset=0.2in,%
            left=1in,right=1in,top=1in,bottom=1in,%
            footskip=.25in]{geometry}

\begin{document}

\newcommand{\Z}{\mathbb{Z}}
\newcommand{\Mod}[1]{\ (\mathrm{mod}\ #1)}

\title{6. Fermat's and Wilson's Theorems}
\section{Exercises}

% 1
\subsection{}
Verify that the theorem is true for $a = 2$ and $p = 5$.\\~\\
$2^{5 - 1} = 16 = 3 * 5 + 1 \equiv 1 \Mod{5}$.

% 2
\subsection{}
Calculate $2^2$ and $20^{10} \Mod{11}$.
\begin{align*}
    2^2 &\equiv 4 \Mod{11}\\
    2^4 &\equiv 5 \Mod{11}\\
    2^8 &\equiv 3 \Mod{11}\\
    2^2 * 2^8 &\equiv 1 \Mod{11}
\end{align*}

% 3
\subsection{}
What are the pairs when $p = 11$?\\~\\
(2, 6), (3, 4), (5, 9), (6, 2), (7, 8).


\section{Problems}

% 1
\subsection{}
What is the least residue of
\begin{equation*}
    5^6 \Mod{6}, \quad 5^8 \Mod{7}, \quad 1945^8 \Mod{7}
\end{equation*}
\begin{gather*}
    5^2 \equiv 4 \Mod{7}\\
    5^4 \equiv 2 \Mod{7}\\
    5^6 \equiv 1 \Mod{7},\\
    5^8 \equiv 4 \Mod{7},\\
    1945 \equiv 545 \equiv 195 \equiv 55 \equiv 6 \Mod{7}\\
    1945^2 \equiv 1 \Mod{7}\\
    1945^8 \equiv 1 \Mod{7}
\end{gather*}

% 2
\subsection{}
What is the least residue of
\begin{equation*}
    5^{10} \Mod{11}, \quad 5^{12} \Mod{11}, \quad 1945^{12} \Mod{11}
\end{equation*}
\begin{gather*}
    5^2 \equiv 3 \Mod{11}\\
    5^4 \equiv 9 \Mod{11}\\
    5^8 \equiv 4 \Mod{11}\\
    5^10 \equiv 1 \Mod{11},\\
    5^12 \equiv 3 \Mod{11},\\
    1945 \equiv 845 \equiv 75 \equiv 9 \Mod{11}\\
    1945^2 \equiv 4 \Mod{11}\\
    1945^4 \equiv 5 \Mod{11}\\
    1945^8 \equiv 3 \Mod{11}\\
    1945^{12} \equiv 4 \Mod{11}
\end{gather*}

% 3
\subsection{}
What is the last digit of $7^{355}$?
\begin{gather*}
    7 \equiv 7 \Mod{10}\\
    7^2 \equiv 9 \Mod{10}\\
    7^4 \equiv 1 \Mod{10}\\
    355 \equiv 35 \equiv 3 \Mod{4}\\
    7^{355} \equiv 7 * 9 \equiv 3 \Mod{10}
\end{gather*}
So, the last digit is 3.

% 4
\subsection{}
What are the last two digits of $7^{355}$?
\begin{gather*}
    7 \equiv 7 \Mod{100}\\
    7^2 \equiv 49 \Mod{100}\\
    7^4 \equiv 1 \Mod{100}\\
    7^{355} \equiv 7 * 49 \equiv 43 \Mod{10}
\end{gather*}

% 5
\subsection{}
What is the remainder when $314^{162}$ is divided by 163?\\
Since 163 is prime, by Fermat's Theorem, $314^{162} \equiv 1 \Mod{p}$.

% 6
\subsection{}
What is the remainder when $314^{162}$ is divided by 7?
\begin{gather*}
    314 \equiv 6 \Mod{7}\\
    314^2 \equiv 1 \Mod{7}\\
    314^162 \equiv 1 \Mod{7}
\end{gather*}

% 7
\subsection{}
What is the remainder when $314^{164}$ is divided by 165?\\
The prime decomposition of 165 is $3 * 5 * 11$.
\begin{gather*}
    314 \equiv 2 \Mod{3}\\
    314^2 \equiv 1 \Mod{3}\\
    314^{164} \equiv 1 \Mod{3},\\
    314 \equiv 4 \Mod{5}\\
    314^2 \equiv 1 \Mod{5}\\
    314^{164} \equiv 1 \Mod{5},\\
    314 \equiv 6 \Mod{11}\\
    314^2 \equiv 3 \Mod{11}\\
    314^4 \equiv 9 \Mod{11}\\
    314^8 \equiv 4 \Mod{11}\\
    314^{16} \equiv 5 \Mod{11}\\
    314^{32} \equiv 3 \Mod{11}\\
    314^{64} \equiv 9 \Mod{11}\\
    314^{128} \equiv 4 \Mod{11}\\
    314^{164} \equiv 4 * 3 * 9 \equiv 9 \Mod{11}.
\end{gather*}
Now have to solve the system of congruences:
\begin{equation*}
    x \equiv 1 \Mod{15}, \quad
    x \equiv 9 \Mod{11}
\end{equation*}
Solving:
\begin{align*}
    15k_1 + 1 &\equiv 9 \Mod{11}\\
    k_1 &\equiv 2 \Mod{11}\\
    x &\equiv 31 \Mod{165}
\end{align*}
So, the remainder is 31.

% 8
\subsection{}
What is the remainder when $2001^{2001}$ is divided by 26?
\begin{gather*}
    2001 \equiv 25 \Mod{26}\\
    2001^2 \equiv 1 \Mod{26}\\
    2001^{2001} \equiv 25 \Mod{26}
\end{gather*}

% 9
\subsection{}
Show that
\begin{equation*}
    (p - 1)(p - 2)...(p - r) \equiv (-1)^rr! \Mod{p}
\end{equation*}
for $r = 1, 2, ..., p - 1$.\\~\\
Expanding the product $(p - 1)(p - 2)...(p - r)$,
notice that all terms will be a multiple of $p$ other than the product
of $-1 * -2 * ... * -r = (-1)^rr!$.

% 10
\subsection{}
\textbf{(a)} Calculate $(n - 1)! \Mod{n}$ for $n = 10, 12, 14,$ and 15.\\
\textbf{(b)} Guess a theorem and prove it.\\~\\
\textbf{(a)} Since $2|9!$ and $5|9!$, then $9! \equiv 0 \Mod{10}$.
Likewise, since 2 and 6 divide 11!, $11! \equiv 0 \Mod{12}$;
since 7 and 2 divide 13!, $13! \equiv 0 \Mod{14}$;
and since 3 and 5 divide 14!, $14! \equiv 0 \Mod{15}$.\\
\textbf{(b)} For any composite, non-square $n$, $n|(n - 1)!$.
If $n$ is composite, can write it as $ab$, with $a, b \in [2, n - 1]$.
Since $a \neq b$, then these $a$ and $b$ must be one of the products of
$(n - 1)! = 1 * 2 * ... * a * ... * b * ... * n - 1$.
The property doesn't hold for square $n = 4$, as $4 \nmid 6$.
Writing $n = a^2$, it clearly holds for $a = 3$, as $3*6|8!$, so $9|8!$.
So it will hold as long as $a$ and $2a$ are products of $(a^2 - 1)!$.
This is true if $2a < a^2 - 1$, or $a^2 - 2a - 1 > 0$.
For $a = 3$, $9 - 6 - 1 > 0$.
The derivative of this function with respect to $a$ is $2a - 2$, which is
non-negative for all $a \geq 0$.
So all $a \geq 3$ will fulfill the equation.

% 11
\subsection{}
Show that $2(p - 3)! + 1 \equiv 0 \Mod{p}$.\\~\\
Equivalently, we want to show that $2(p - 3)! \equiv -1 \Mod{p}$.
From Wilson's Theorem, we know that $(p - 1)! \equiv -1 \Mod{p}$,
so if we show that $2(p - 3)! \equiv (p - 1)! \Mod{p}$, we will be done.
Notice that $(p - 1)! = (p - 1)(p - 2)(p - 3)!$.
$(p - 1)(p - 2) \equiv 2 \Mod{p}$, so
$(p - 1)(p - 2)(p - 3)! \equiv 2(p - 3)! \Mod{p}$.

% 12
\subsection{}
In 1732 Euler wrote: "I derived [certain] results from the elegant theorem, of whose
truth I am certain, although I have no proof: $a^n - b^n$ is divisible by the prime
$n + 1$ if neither $a$ nor $b$ is." Prove this theorem, using Fermat's Theorem.\\~\\
Since $n + 1 \nmid a$ and $n + 1 \nmid b$, $(a, n + 1) = 1$ and $(b, n + 1) = 1$.
We can thus use Fermat's Theorem to get
$a^n \equiv 1 \Mod{n+1}$ and $b^n \equiv 1 \Mod{n+1}$.
Subtracting these two congruences, have $a^n - b^n \equiv 0 \Mod{n+1}$,
or, equivalently, $n + 1 \mid a^n - b^n$.

% 13
\subsection{}
Note that
\begin{align*}
    6! &\equiv -1 \Mod{7},\\
    5!1! &\equiv 1 \Mod{7},\\
    4!2! &\equiv -1 \Mod{7},\\
    3!3! &\equiv 1 \Mod{7}.
\end{align*}
Try the same sort of calculation $\Mod{11}$.\\~\\
$10! \equiv -1 \Mod{11}$ by Wilson's theorem. For the rest:
\begin{align*}
    9!1! &\equiv 1 \Mod{11},\\
    8!2! &\equiv -1 \Mod{11},\\
    7!3! &\equiv 1 \Mod{11},\\
    6!4! &\equiv -1 \Mod{11},\\
    5!5! &\equiv 1 \Mod{11}.\\
\end{align*}

% 14
\subsection{}
Guess a theorem from the data of Problem 13, and prove it.\\~\\
If $p$ is prime, for $k \in [2, p - 1]$, $(p - k)!(k - 1)! \equiv -1^k \Mod{p}$.\\
$(p - 1)! = (p - 1) * (p - 2) ... * (p - (k - 1)) * (p - k)!$.
From the product up to the $(p - k)!$ term we can cancel all factors of $p$,
leaving us with $-1 * -2 * ... * -(k - 1) = -1^{k - 1} * (k - 1)!$.
So, we know that $(p - 1)! \equiv -1^{k - 1} * (k - 1)! \equiv -1 \Mod{p}$.
Multiplying both sides by the $-1^{k - 1}$ term, have
 $(p - k)!(k - 1)! \equiv -1^k \Mod{p}$.

% 15
\subsection{}
Suppose that $p$ is an odd prime.\\
\textbf{(a)} Show that
\begin{equation*}
    1^{p - 1} + 2^{p - 1} + ... + (p - 1)^{p - 1} \equiv -1 \Mod{p}
\end{equation*}
\textbf{(b)} Show that
\begin{equation*}
    1^p + 2^p + ... + (p - 1)^p \equiv 0 \Mod{p}
\end{equation*}
\textbf{(a)} 
By Fermat's theorem, we have $a^{p - 1} \equiv 1 \Mod{p}$ if $(a, p) = 1$.
All $a \in [1, p - 1]$ will have $(a, p) = 1$.
So, the equation is equivalent to $1 + 1 + ... + 1 = p - 1 \equiv -1 \Mod{p}$.\\
\textbf{(b)} Fermat's Theorem can alternatively be stated $a^p \equiv a \Mod{p}$.
So, our equation becomes $1 + 2 + ... + (p - 1)$.
Notice that this can be rearranged into $p / 2$ pairs of $a$, $p - a$ so that
the $a$s cancel out, leaving only terms of $p$.
So the sum must be $\equiv 0 \Mod{p}$.

% 16
\subsection{}
Show that the converse of Fermat's Theorem is false.
[Broad hint: consider $2^{340}\Mod{341}$.]\\~\\
Write $341 = 11 * 31$ and find the residual of $2^{340}$ for both factors.
\begin{gather*}
    2^4 \equiv 5 \Mod{11}\\
    2^8 \equiv 3 \Mod{11}\\
    2^{16} \equiv 9 \Mod{11}\\
    2^{32} \equiv 4 \Mod{11}\\
    2^{64} \equiv 5 \Mod{11}\\
    2^{128} \equiv 3 \Mod{11}\\
    2^{256} \equiv 9 \Mod{11}\\
    2^{340} = 2^{256 + 64 + 16 + 4} \equiv 1 \Mod{11}\\
    2^5 \equiv 1 \Mod{31}\\
    2^{340} \equiv 1 \Mod{31}
\end{gather*}
So, $2^{340}$ can be written as $(11 * 31)t + 1$, implying $2^{340} \equiv 1 \Mod{341}$
with 341 composite.

% 17
\subsection{}
Show that for any two different primes $p$, $q$,\\
\textbf{(a)} $pq|(a^{p+q} - a^{p+1} - a^{q+1} + a^2)$ for all $a$.\\
\textbf{(b)} $pq|(a^{pq} - a^p - a^q + a)$ for all $a$.\\~\\
\textbf{(a)}
Write $a^{p+q} - a^{p+1} - a^{q+1} + a^2 = (a^p - a)(a^q - a)$.
Note that each of these terms can be written $a(a^{x - 1} - 1)$.
Since the $x$ in both is prime, by Fermat's Theorem $x|a^{x - 1} - 1$.
In other words, $p|(a^p - a)$ and $q|(a^q - a)$, so $pq$ divides the whole thing.\\
\textbf{(b)}
Fermat's Theorem can be stated $a^p \equiv a \Mod{p}$.
So, $(a^q)^p \equiv a^q \Mod{p}$, and $a^{pq} - a^p \equiv a^q - a \Mod{p}$.
Likewise, $a^{pq} - a^q \equiv a^p - a \Mod{q}$.
So, subtracting the residue in either equation yields $a^{pq} - a^p - a^q + a$,
which is divisible by both $p$ and $q$.

% 18
\subsection{}
Show that if $p$ is an odd prime, then $2p|(2^{2p-1} - 2)$.\\~\\
Can write $2^{2p-1} - 2 = 2(2^{2p-2} - 1) = 2(2^{p-1} - 1)(2^{p-1} + 1)$.
We want to show that $p|(2^{p-1} - 1)(2^{p-1} + 1)$.
Since $(p, 2) = 1$, we can use Fermat's Theorem to show that $p|(2^{p-1} - 1)$,
so $2p|(2^{2p-1} - 2)$.

% 19
\subsection{}
For what $n$ is it true that
\begin{equation*}
    p|(1 + n + n^2 + ... + n^{p - 2}) \; ?
\end{equation*}
If $p|n$ then we would need $p|1$ for the above to hold, so that is not a viable condition.
The geometric series can be written $\frac{n^{p - 1} - 1}{n - 1}$.
As long as $(p, n) = 1$, we know by Fermat's Theorem that $p|(n^{p - 1} - 1)$.
We're not sure whether $p$ divides the whole fraction if the bottom fraction is
divisble by $p$. So, we should also exclude $n \equiv 1 \Mod{p}$ from viable $n$.

% 20
\subsection{}
Show that every odd prime except 5 divides some number of the form $111...11$
($k$ digits, all ones).\\~\\
In the previous exercise, we showed $p|(1 + n + n^2 + ... + n^{p - 2})$
if $n \not\equiv 0$ or $1 \Mod{p}$.
Observe that numbers of the form $111...1$ can be written as such a geometric
series when $n = 10$.
Since $p$ is a prime other than 2 and 5, $10 \not\equiv 0 \Mod{p}$.
For $p = 3$, have $3|111$.
For $p = 7$, we know $10 \equiv 3 \Mod{7}$, so $\not\equiv 1$.
All other primes are $> 10$, and thus be $\equiv 10 \Mod{p}$, so $\not\equiv 1$.


\end{document}
EOF
