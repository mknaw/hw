\documentclass{article}
\usepackage{amsmath}
\usepackage{amssymb}
\usepackage[a4paper,bindingoffset=0.2in,%
            left=1in,right=1in,top=1in,bottom=1in,%
            footskip=.25in]{geometry}

\begin{document}

\newcommand{\Z}{\mathbb{Z}}

\title{1. Integers}
\section{Exercises}

% 1
\subsection{}
Which integers divide zero? \\~\\
Using the definition, all $a \in \Z$ for which $\exists d \in \Z$ that satisfies $ad = 0$ divide 0.
This condition holds for any $a$ when $d = 0$, so all integers divide 0.

% 2
\subsection{}
Show that if $a|b$ and $b|c$, then $a|c$. \\~\\
$a|b \Rightarrow \exists m \in \Z \ni am = b$, $b|c \Rightarrow \exists n \in \Z \ni bn = c$.
Rewriting, we have $amn = c$. $\therefore mn$ is an integer that divides $c$.

% 3
\subsection{}
Prove that if $d|a$ then $d|ca$ for any integer c. \\~\\
$a = dq \Rightarrow ca = cdq$, with $cq \in \Z$.

% 4
\subsection{}
What are (4, 14), (5, 15) and (6, 16)? \\~\\
2, 5, and 2.

% 5
\subsection{}
What is ($n$, 1), where $n$ is any positive integer? What is ($n$, 0)? \\~\\
If $d = (n, 1)$, we have by (i) that $d|n$ and $d|1$.
But, the only $d$ that satisfies $d|1$ is 1, so $d = 1$. \\
If $d = (n, 0)$, we have by (i) that $d|n$ and $d|0$.
We know from the first exercise that all $d \in \Z$ divide 0, so this is not helpful.
So, it must be the greatest possible $d$ that divides $n$, which is $n$.


% 6
\subsection{}
If $d$ is a positive integer, what is $(d, nd)$? \\~\\
$d$ divides $nd$, since $\exists q \ni qd = nd$, namely $q = n$.
It also obviously divides $d$, since $1d = d$ (i).
We cannot have a $c \in \Z \ni c > d$ for which the first condition holds,
namely that $c|d$ (for nonzero $d$, $\nexists b \in \Z \ni bc = d$, only non-integer $b \in (0, 1)$) (ii).

% 7
\subsection{}
What are $q$ and $r$ if $a = 75$ and $b = 24$? If $a = 75$ and $b = 25$?\\~\\
$75 = 3 * 24 + 3$ ($q = 3$, $r = 3$). $75 = 3 * 25 + 0$ ($q = 3$, $r = 0$).

% 8
\subsection{}
Verify that $a = bq + r \Rightarrow (a, b) = (b, r)$ when $a = 16$, $b = 6$, and $q = 2$.\\~\\
$(a, b) = 2$, $(b, r) = 2$.

% 9
\subsection{}
Calculate (343, 280) and (578, 442).\\~\\
\begin{align}
  343 &= 1 * 280 + 63 \\
  280 &= 4 * 63 + 28 \\
  63 &= 2 * 28 + 7 \\
  28 &= 4 * 7 + 0 \\
  (343, 280) &= 7
\end{align}

\begin{align}
  578 &= 1 * 442 + 136 \\
  442 &= 3 * 136 + 34 \\
  136 &= 4 * 34 + 0 \\
  (578, 442) &= 34
\end{align}

\section{Problems}

% 1
\subsection{}
Calculate (314, 159) and (4144, 7696).
\begin{align}
  314 &= 1 * 159 + 155  \\
  159 &= 1 * 155 + 4 \\
  155 &= 38 * 4 + 3 \\
  4 &= 1 * 3 + 1 \\
  3 &= 3 * 1 + 0 \\
  (314, 159) &= 1
\end{align}

\begin{align}
  7696 &= 1 * 4144 + 3552 \\
  4144 &= 1 * 3552 + 592 \\
  3552 &= 6 * 592 + 0 \\
  (4414, 7696) &= 592
\end{align}

% 2
\subsection{}
Calculate (3141, 1592) and (10001, 100083).
\begin{align}
  3141 &= 1 * 1592 + 1549  \\
  1592 &= 1 * 1549 + 43 \\
  1549 &= 36 * 43 + 1 \\
  43 &= 43 * 1 + 0 \\
  (3141, 1592) &= 1
\end{align}

\begin{align}
  100083 &= 10 * 10001 + 73 \\
  10001 &= 137 * 73 + 0 \\
  (10001, 100083) &= 73
\end{align}

% 3
\subsection{}
Find x and y such that 314x + 159y = 1.
\begin{align}
  1 &= 4 - 1 * 3 \\
  1 &= 4 - 1 * (155 - 38 * 4) \\
  1 &= 39 * 4 - 155 \\
  1 &= 39 * (159 - 155) - 155 \\
  1 &= 39 * 159 - 40 * 155 \\
  1 &= 39 * 159 - 40 * (314 - 159) \\
  1 &= 314 * (-40) + 159 * 79
\end{align}

% 4
\subsection{}
Find x and y such that 4144x + 7696y = 592.
\begin{align}
  592 &= 4144 - 1 * 3552 \\
  592 &= 4144 - 1 * (7696 - 4144) \\
  592 &= 4144 * 2 - 7696 * 1
\end{align}

% 5
\subsection{}
If $N = abc + 1$, prove that $(N, a) = (N, b) = (N, c) = 1$. \\~\\
1 is certainly a divisor of $N$, $a$, $b$, and $c$, so the only thing left to show
is that 1 is the greatest divisor of each.
Assume $\exists k \ni k > 1, k|N$ and $k|a$.
Plug in the definition of $N$ to get $k|abc + 1$.
Then $\exists y \ni ky =a$, $\exists x \ni kx = abc + 1$.
Combine the two to get $kx = kybc + 1 \Rightarrow k(x - ybc) = 1$.
For $k, x, y, b, c \in \Z$, only $k = 1, x - ybc = 1$ or $k = -1, x - ybc = -1$
fulfill the equation.
Since both $k = -1$ and $k = 1$ contradict the assumption that $k > 1$,
1 must be the greatest divisor of $(N, a)$.
One could analogously show this property for $b$ and $c$.

% 6
\subsection{}
Find two different solutions of 299x + 247y = 13.
\begin{align}
  299 &= 1 * 247 + 52  \\
  247 &= 4 * 52 + 39 \\
  52 &= 1 * 39 + 13
\end{align}
\begin{align}
  13 &= 52 - 39 \\
  13 &= 52 - (247 - 4 * 52) \\
  13 &= 5 * 52 - 247 \\
  13 &= 5 * (299 - 247) - 247 \\
  13 &= 5 * 299 - 6 * 247
\end{align}
Observe that $13 = 52 - 39 = 4 * 13 - 3 * 13$.
Pick $a, b \ni 3a - 4b = 1$, or $a = \frac{1 + 4b}{3}$, for example $a = 3$, $b = 2$.
Then
\begin{align}
  13 &= 3 * 3 * 13 - 2 * 4 * 13 \\
  13 &= 3 * 39 - 2 * 52 \\
  13 &= 3 * (247 - 4 * 52) - 2 * 52 \\
  13 &= 3 * 247 - 14 * 52 \\
  13 &= 3 * 247 - 14 * (299 - 247) \\
  13 &= 17 * 247 - 14 * 299 \\
\end{align}

% 7
\subsection{}
Prove that if $a|b$ and $b|a$ then $a = b$ or $a = -b$. \\~\\
If $a|b$ then $\exists x \in \Z \ni ax = b$.
If $b|a$ then $\exists y \in \Z \ni by = a$.
Combining, have $bxy = b \Rightarrow xy = 1$.
The only two solutions to this equation in the domain of integers
are $(1, 1)$ and $(-1, -1)$.
So, $ax = b \Rightarrow a = b$ when $x = 1$, $a = -b$ when $x = -1$.

% 8
\subsection{}
Prove that if $a|b$ and $a > 0$, then $(a, b) = a$. \\~\\
$a$ is a divisor of $a$ since $1a = a$, and $b$ since it is given.
It remains to show that $a$ is the greatest common divisor of $a$ and $b$.
Assume $\exists c \ni c > a > 0, c|a, c|b$.
Then $\exists x \ni cx = a$. Since $c$ and $a$ are both greater than 0,
we need $x > 0$ for the equality to hold. But even at $x = 1$, the smallest
$x \in \Z \ni x > 0$, $c > a$. So there are no such $x$ or $c$.

% 9
\subsection{}
Prove that $((a, b), b) = (a, b)$. \\~\\
By the definition of $(x, y) = d$, $d|x$ and $d|y$.
So, $(a, b)|b$.
Now we must show that $(a, b)$ is the greatest common divisor of $(a, b)$ and $b$.
Assume $\exists c \ni c > (a, b), c|(a, b), c|b$.
Similarly to exercise 8, if $(a, b) > 0$, can't have an $x \in \Z^+ \ni cx = (a, b)$.\\
Let $(a, b) = d \ni d < 0$.
So $\exists y, z \in \Z \ni y * d = a, z * d = b$.
But then $-y * -d = a$, $-z * -d = b$, so $d$ is a divisor of both $a$ and $b$
and $-d > d$ and $(a, b) \neq d$.

% 10
\subsection{}
\textbf{(a)} Prove that $(n, n + 1) = 1$ for all $n > 0$.\\
\textbf{(b)} If $n > 0$, what can $(n, n + 2)$ be? \\~\\
\textbf{(a)} $1|n$ and $1|n + 1$ since $\forall x \in \Z, 1x = x$.
Suppose $\exists k > 1 \ni k|n, k|n + 1$.
So $\exists x \ni kx = n$ and $\exists y \ni ky = n + 1$.
Combine the two to get $ky = kx + 1 \Rightarrow k(y - x) = 1$.
The only viable pairs of $k, (y - x) \in \Z$ for which the equality holds are
$(1, 1)$ and $(-1, -1)$, both of which violate the assumption that $k > 1$.\\
\textbf{(b)} Similarly to above, $\exists k \ni k|n, k|n + 2 \Rightarrow
\exists x, y \ni kx = n, ky = n + 2$.
So $ky = kx + 2 \Rightarrow k(y - x) = 2$.
Since GCDs must be positive, as shown in the previous exercise, we can have
$k = 1$ when $(y - x) = 2$ or $k = 2$ when $(y - x) = 1$.

% 11
\subsection{}
\textbf{(a)} Prove that $(k, n + k) = 1$ iff $(k, n) = 1$.\\
\textbf{(b)} Is it true that $(k, n + k) = d$ iff $(k, n) = d$?\\~\\
\textbf{(a)} First, we show $(k, n) = 1 \Rightarrow (k, n + k) = 1$.
Let $(k, n + k) = d \ni d > 1$.
By the definition of the GCD, we know $d|k, d|n+k$.
So $\exists a, b \ni ad = k, bd = n + k$.
This means $d(b - a) = n$, implying $d|n$.
From the given $(k, n) = 1$, we know that any $\forall c, c|k, c|n \Rightarrow c \leq 1$,
so we have a contradiction.\\
Next, we show $(k, n + k) = 1 \Rightarrow (k, n) = 1$.
By \textbf{Lemma 1}, $\forall c \ni c|k, c|n+k \Rightarrow c|k+n+k$ and $(k, n + k) = 1 \Rightarrow c \leq 1$.
Let $(k, n) = d \ni d > 1$.
By the GCD defintion, $d|k$, and by an application of \textbf{Lemma 1}, $d|n+k$.
Applying \textbf{Lemma 1} again, $d|k, d|n+k \Rightarrow d|k+n+k$.
But, we've shown that such $d \leq 1$, so we have a contradiction.\\
\textbf{(b)} First, show $(k, n) = d \Rightarrow (k, n + k) = d$.
Assume $\exists q \ni q|k, q|n + k, q > d$.
Like in \textbf{(a)}, we know that $q|k, q|n+k \Rightarrow q|n$.
This contradicts $(k, n) = d$.\\
Now show $(k, n + k) = d \Rightarrow (k, n) = d$.
Like in \textbf{(a)}, we know that $d|k, d|n+k \Rightarrow d|n$.
So $d$ is definitely a divisor for $n$ and $k$.
Assume $\exists c \ni c|n, c|k, c > d$.
$c|n, c|k \Rightarrow c|n+k$, which would contradict $(k, n+k) = d$.\\
So yes, the statement holds.

% 12
\subsection{}
Prove: If $a|b$ and $c|d$, then $ac|bd$.\\~\\
$\exists x, y \ni ax = b, cy = d$.
So $bd = (ac)(xy)$ and $ac|bd$.

% 13
\subsection{}
Prove: If $d|a$ and $d|b$, then $d^2|ab$.\\~\\
$\exists x, y \ni dx = a, dy = b$.
So $ab = d^2xy$ and $d^2|ab$.

% 14
\subsection{}
Prove: If $c|ab$ and $(c, a) = d$, then $c|db$.\\~\\
Since $(c, a) = d$, we can write $a = dx, c = dy \ni x, y \leq d$.
Furthermore, $(x, y) = 1$ because if there were a
$z > 1 \ni z|x, z|y$, then $dz|a$ and $dz|c$ with $dz > d$, violating the GCD definition.
We rewrite $ab = dxb \ni c|dxb$.
Subbing in the new definition for $c$, $dy|dxb$, i.e. $y|xb$.
By \textbf{Corollary 1}, if $y|xb$ and $(x, y) = 1$, then $y|b$.
So $dy|db$, i.e. $c|db$.

% 15
\subsection{}
\textbf{(a)} If $x^2 + ax + b = 0$ has an integer root, show that it divides $b$.\\
\textbf{(b)} If $x^2 + ax + b = 0$ has a rational root, show that it is in fact an integer.\\~\\
\textbf{(a)} Assuming $a$ and $b$ are integers.
$b = -x^2 - ax = x(-x - a)$.
Since $a$, $x$ are both integers, their linear combinations are integers as well.
So $\exists q \in \Z \ni bq = x$, namely $q = -x - a$.\\
\textbf{(b)} If $x$ is rational, it can be written as $\frac{n}{k} \ni n \in \Z, k \in \Z^+$
and $n, k$ are relatively prime.
% Also know $k|n \Rightarrow \exists q \in \Z \ni qk = n$, meaning $\frac{n}{k} = q \in \Z$.
Subbing in, have $(\frac{n}{k})^2 + a\frac{n}{k} + b = 0$.
Can rewrite this as $-k(bk + an) = n^2$.
This implies $k|n^2$.
Since $n, k$ are relatively prime, $(n, k) = 1$.
Using the result from exercise \textbf{14}, $k|nn$ and $(n, k) = 1 \Rightarrow k|1n$.
If $k|n$ and $k|k$ and $(n, k) = 1$, then $k = 1$ and $\frac{n}{k} = n$, with $n \in \Z$.

\end{document}
EOF
