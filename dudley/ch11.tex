\documentclass{article} \usepackage{amsmath}
\usepackage{amssymb}
\usepackage[a4paper,bindingoffset=0.2in,%
            left=1in,right=1in,top=1in,bottom=1in,%
            footskip=.25in]{geometry}

\begin{document}

\newcommand{\Z}{\mathbb{Z}}
\newcommand{\s}{\sigma}
\newcommand{\p}{\phi}
\newcommand{\e}{\equiv}
\newcommand{\m}[1]{\ (\mathrm{mod}\ #1)}

\title{11. Quadratic Congruences}
\section{Exercises}

% 1
\subsection{}
Convert $2x^2 + 3x + 1 \e 0 \m{5}$ to a quadratic congruence
whose first coefficient is 1.\\~\\
Since $3*2 \e 1 \m{5}$, the congruence is equivalent to
$x^2 + 4x + 3 \e 0 \m{5}$.

% 2
\subsection{}
Change the quadratic in Exercise 1 to the form (3).\\~\\
$(x + 2)^2 \e 1 \m{5}$.

% 3
\subsection{}
By inspection, find all the solutions of the congruence in Exercise 2.\\~\\
$y^2 \e 1 \m{5}$ has solutions 1 and 4, so the congruence in Exercise 2
has solutions 2 and 4.

% 4
\subsection{}
If $p > 3$, what are the two solutions of $x^2 \e 4 \m{p}$?\\~\\
2 and $p - 2$.

% 5
\subsection{}
For what values of $a$ does $x^2 \e a \m{7}$ have two solutions?\\~\\
1, 4, 2.

% 6
\subsection{}
Find the solutions of $x^2 \e 8 \m{31}$.\\~\\
$8 \e 39 \e 70 \e 101 \e 132 \e 2^2 * 33 \e 2^2 * 2 \m{31}$.\\
$2 \e 33 \e 64 \e 8^2 \m{31}$.\\
$2^2 * 8^2 \e 16^2 \m{31}$.\\
15 and 16 are solutions to $x^2 \e 8 \m{31}$.

% 7
\subsection{}
What is $(1/3)$? $(1/7)$? $(1/11)$? In general, what is $(1/p)$?\\~\\
All of them must be 1, since $1$ and $p - 1$ are solutions to the congruences.

% 8
\subsection{}
What is $(4/5)$? $(4/7)$? $(4/p)$ for any odd prime $p$?\\~\\
For $p > 3$, it should be 1, as $2$, $p - 2$ are both solutions.

% 9
\subsection{}
Induce a theorem from the two preceding exercises.\\~\\
$(k^2/p) = 1$ as long as $p \nmid k$.

% 10
\subsection{}
Verify that if $(a/p) = -1$ and $a \e b \m{p}$, then $(b/p) = -1$.\\~\\
If $(a/p) = -1$, then $\not\exists x \ni x^2 \e a \e b \m{p}$.

% 11
\subsection{}
Prove (B), using (6).\\~\\
$(a^2/p) \e a^{2(p-1)/2} \e a^{p-1}$.
By Fermat's Theorem, $a^{p-1} \e 1 \m{p}$.
So, $(a^2/p) \e 1 \m{p}$, and since $(a^2/p) = 1$ or -1, $(a^2/p) = 1$.

% 12
\subsection{}
Prove that $(4a/p) = (a/p)$.\\~\\
Know that $(4a/p) = (4/p)(a/p)$.
Also know that $p \nmid 4$, as $p$ is an odd prime.
So we can use the result from the previous, that $(2^2/p) = 1$.

% 13
\subsection{}
Evaluate $(19/5)$ and $(-9/13)$ by using (A) and (B).\\~\\
$(19/5) = (4/5) = (2^2/5) = 1$.
$(-9/13) = (4/13) = (2^2/13) = 1$.

% 14
\subsection{}
For which of the primes 3, 5, 7, 11, 13, 17, 19, and 23
is -1 a quadratic residue?\\~\\
-1 is a quadratic residue for 5, 13, and 17.

% 15
\subsection{}
Evaluate $(6/7)$ and $(2/23)(11/23)$.\\~\\
$(6/7) = (-1/7) = -1$.
$(2/23)(11/23) = (23/2)*-(23/11) = -(1/2)(1/11) = -1$.

\section{Problems}

% 1
\subsection{}
Which of the following congruences have solutions?
\begin{equation*}
    x^2 \e 7 \m{53},\quad
    x^2 \e 14 \m{31},\quad
    x^2 \e 53 \m{7},\quad
    x^2 \e 25 \m{997}.
\end{equation*}
$(7/53) = (53/7) = (4/7) = 1$.\\
$(14/31) = (2/31)(7/31) = -(31/2)(31/7) = -(1/2)(3/7) = 1$.\\
$(53/7) = (4/7) = 1$.\\
$(25/997) = 1$.\\

% 2
\subsection{}
Which of the following congruences have solutions?
\begin{equation*}
    x^2 \e 8 \m{53},\quad
    x^2 \e 15 \m{31},\quad
    x^2 \e 54 \m{7},\quad
    x^2 \e 625 \m{9973}.
\end{equation*}
$(8/53) = (2/53)(4/53) = (2/53) = -1$.\\
$(15/31) = (3/31)(5/31) = -(31/3)(31/5) = -(1/3)(1/5) = -1$.\\
$(54/7) = (5/7) = -1$.\\
$(625/9973) = (25/9973)(25/9973) = 1$.\\

% 3
\subsection{}
Find solutions for the congruences in Problem 1 that have them.\\~\\
$x^2 \e 7 \e 60 \e 2^2*15 \m{53}$.\\
$15 \e 68 \e 121 \e 11^2 \m{53}$.\\
So, 22 and 31 are solutions to the congruence.\\
$x^2 \e 14 \e 169 \e 13^2 \m{31}$.\\
So, 13 and 18 are solutions to the congruence.\\
$x^2 \e 53 \e 4 \e 2^2 \m{7}$.\\
So, 2 and 5 are solutions to the congruence.\\
$x^2 \e 25 \m{997}$.\\
5 and 992 are solutions to the congruence.

% 4
\subsection{}
Find solutions for the congruences in Problem 2 that have them.\\~\\
$x^2 \e 625 \m{9973}$.\\
25 and 9948 are solutions to the congruence.

% 5
\subsection{}
Calculate $(33/71)$, $(34/71)$, $(35/71)$, and $(36/71)$.\\~\\
$(33/71) = (71/33) = (5/33) = (33/5) = (3/5) = -1$.\\
$(33/71) = (2/71)(17/71) = (71/17) = (3/17) = (17/3) = (2/3) = -1$.\\
$(35/71) = (5/71)(7/71) = -(71/5)(71/7) = -(1/5)(1/7) = -1$.\\
$(36/71) = 1$.\\

% 6
\subsection{}
Calculate $(33/73)$, $(34/73)$, $(35/73)$, and $(36/73)$.\\~\\
$(33/73) = (3/73)(11/73) = (73/3)(73/11) = (1/3)(7/11) = -(11/7) = -(4/7) = -1$.\\
$(34/73) = (2/73)(17/73) = (73/17) = (5/17) = (17/5) = (2/5) = -1$.\\
$(35/73) = (5/73)(7/73) = (3/5)(3/7) = -(2/3)(1/3) = 1$.\\
$(36/73) = 1$.\\

% 7
\subsection{}
Solve $2x^2 + 3x + 1 \e 0 \m{7}$ and $2x^2 + 3x + 1 \e 0 \m{101}$.
\begin{gather*}
    x^2 + 12x + 4 \e 0 \m{7}\\
    (x + 6)^2 \e 4 \m{7}\\
    x \in \{3, 6\},\\
    x^2 + 52x + 51 \e 0 \m{101},\\
    (x + 26)^2 \e 625 \m{101},\\
    x \in \{50, 100\}
\end{gather*}

% 8
\subsection{}
Solve $3x^2 + x + 8 \e 0 \m{11}$ and $3x^2 + x + 52 \e 0 \m{11}$.
\begin{gather*}
    x^2 + 4x + 10 \e 0 \m{11}\\
    (x + 2)^2 \e 16 \m{11}\\
    x \in \{2, 5\},\\
    x^2 + 4x + 10 \e 0 \m{11}\\
    x \in \{2, 5\}
\end{gather*}

% 9
\subsection{}
Calculate $(1234/4567)$ and $(4321/4567)$.
\begin{gather*}
    (1234/4567) = (2/4567)(617/4567) = (4567/617) = (248/617)\\
    = (2/617)(4/617)(31/617) = (31/617) = (617/31) = (28/31)\\
    = (4/31)(7/31) = -(31/7) = -(3/7) = 1\\
    (4321/4567) = (29/4567)(149/4567) = (4567/29)(4567/149)\\
    = (14/29)(97/149) = (2/29)(7/29)(149/97) = -(29/7)(52/97)\\
    = -(1/7)(4/97)(13/97) = -(97/13) = -(6/13) = 1
\end{gather*}

% 10
\subsection{}
Calculate $(1356/4567)$ and $(6531/4567)$.\\~\\
\begin{gather*}
    (1356/4567) = (3/4567)(4/4567)(113/4567) = -(4567/3)(4567/113)\\
    = -(1/3)(47/113) = -(113/47) = -(19/47) = (47/19) = (9/19) = 1\\
    (6531/4567) = (3/4567)(7/4567)(311/4567) = -(3/7)(213/311)\\
    = (3/311)(71/311) = (2/3)(27/71) = -(3/71) = (2/3) = -1
\end{gather*}

% 11
\subsection{}
Show that if $p = q + 4a$ ($p$ and $q$ are odd primes),
then $(p/q) = (a/q)$.\\~\\
Know $p \e 4a \m{q}$, so $(p/q) = (4a/q) = (4/q)(a/q) = (a/q)$.

% 12
\subsection{}
Show that if $p = 12k + 1$ for some $k$, then $(3/p) = 1$.\\~\\
Know $p \e 1 \m{3}$, so $(p/3) = (1/3) = 1$.
Since $p \e 1 \m{4}$, know that $(p/3) = (3/p)$, so $(3/p) = 1$.

% 13
\subsection{}
Show that Theorem 6 could also be written $(2/p) = (-1)^{(p^2-1)/8}$
for odd primes $p$.\\~\\
If $p \e r \m{8}$, then $p = 8k + r$, $p^2 = 64k^2 + 16k + r^2$.
The first two terms divided by 8 are $8k^2 + 2k$, which is always even.
So we only have to examine $(r^2 - 1)/8 \m{2}$ for the following.
If $p \e 1 \m{8}$, then $(1 - 1)/8 \e 0 \m{2}$.
If $p \e 7 \m{8}$, then $(49 - 1)/8 \e 0 \m{2}$.
Since -1 to an even power is 1, and $(2/p) = 1$ if $p \e 1$ or $7 \m{8}$,
the equation holds for $p \ni (2/p) = 1$.
If $p \e 3 \m{8}$, then $(9 - 1)/8 \e 1 \m{2}$.
If $p \e 5 \m{8}$, then $(25 - 1)/8 \e 1 \m{2}$.
Since -1 to an odd power is -1, and $(2/p) = -1$ if $p \e 3$ or $5 \m{8}$,
the equation holds for $p \ni (2/p) = -1$.

% 14
\subsection{}
Show that the quadratic reciprocity theorem could also be written
$(p/q)(q/p) = (-1)^{(p-1)(q-1)/4}$ for odd primes $p$ and $q$.\\~\\
The quadratic reciprocity theorem states that if $p \e q \e 3 \m{4}$,
then $(p/q) = -(q/p)$. Otherwise, $(p/q) = (q/p)$.
Since $p$, $q$ are odd primes, each can either be $\e 1$ or $\e 3 \m{4}$.
If both $p, q \e 3 \m{4}$, then $(4k_p + 2)(4k_q + 2)/4 =
(16k_pk_1 + 8k_p + 8k_q + 4)/4$.
Thus the exponent is odd, and -1 raised to that exponent is -1.
We also know that if $p \e q \e 3 \m{4}$, then $(p/q) = -(q/p)$,
so $(p/q)(q/p) = -((q/p)^2) = -1$.
Now consider both $p, q \e 1 \m{4}$.
Then $(p-1)(q-1)/4 = 4k_p4k_q/4$ for some $k_p$, $k_q$, and the exponent is even.
Without loss of generality, if $p \e 1$ and $q \e 3 \m{4}$,
then $(p-1)(q-1)/4 = 4k_p(4k_q + 2)/4$, and the exponent is even.
If the exponent is even, then -1 raised to that exponent is 1.
We know that if either $p$ or $q \not\e 3 \m{4}$, then $(p/q) = (q/p)$,
so $(p/q)(q/p) = (q/p)^2 = 1$.

% 15
\subsection{}
Student A says, "I've checked all the way up to 100 and I still haven't found
$n$ so that $n^2 + 1$ is divisible by 7. I'm tired now - I'll find one tomorrow."
Student B says, after a few seconds of reflection, "No you won't."
How did B know so quickly?\\~\\
This problem can be rewritten as $x^2 \e 6 \m{7}$.
Can examine the first few integers (under 7) to see that 6 is a nonresidue
$\m{7}$: $1^2 \e 1$, $2^2 \e 4$, $3^2 \e 2 \m{7}$.

% 16
\subsection{}
Show that if $a$ is a quadratic residue $\m{p}$ and $ab \e 1 \m{p}$,
then $b$ is a quadratic residue $\m{p}$.\\~\\
It follows that $(a/p) = 1$.
We know that $(1/p) = 1$ for any $p$.
So, since $ab \e 1 \m{p}$, $(ab/p) = (1/p) = 1$.
Also know that $p \nmid a, b$, since if it did divide either, $ab \e 0 \m{p}$.
Then $(ab/p) = (a/p)(b/p)$, and $(b/p) = 1$.

% 17
\subsection{}
Does $x^2 \e 211 \m{159}$ have a solution? Note that 159 is not prime.\\~\\
Yes, by inspection $211 + 2 * 159 = 529 = 23^2$ is a solution.

% 18
\subsection{}
Prove that if $p \e 3 \m{8}$ and $(p - 1)/2$ is prime,
then $(p - 1)/2$ is a quadratic residue $\m{p}$.\\~\\
Since $\frac{p-1}{2}$ is prime, we can use the quadratic reciprocity theorem.
If $p \e 3 \m{8}$, then $p \e 3 \m{4}$.
But, if we write $p = 8k + 3$, we see that
$\frac{p - 1}{2} = \frac{8k + 3 - 1}{2} = 4k + 1 \e 1 \m{4}$.
So, $(\frac{p-1}{2}/p) = (p/\frac{p-1}{2})$.
$p \e p - 2(\frac{p-1}{2}) \e 1 \m{\frac{p-1}{2}}$,
so $(p/\frac{p-1}{2}) = (1/\frac{p-1}{2}) = 1$.

% 19
\subsection{}
Generalize Problem 16 by finding what condition on $r$ will guarantee that if
$a$ is a quadratic residue $\m{p}$ and $ab \e r \m{p}$, then $b$ is a quadratic
residue $\m{p}$.\\~\\
It follows that $(a/p) = 1$.
Since $ab \e r \m{p}$, $(ab/p) = (r/p)$.
Since $p \nmid a, b$, $(ab/p) = (a/p)(b/p) = (b/p) = (r/p)$.
So, $b$ a quadratic residue if, and only if, $r$ a quadratic residue.

% 20
\subsection{}
Suppose that $p = q + 4a$, where $p$ and $q$ are odd primes.
Show that $(a/p) = (a/q)$.\\~\\
Know that $p \e 4a \m{q}$, so $(p/q) = (4a/q) = (a/q)$.
Conversely, $q \e -4a \m{p}$, so $(q/p) = (-1/p)(4a/p) = (-1/p)(a/p)$.
Know that if $p \e 1 \m{4}$, then $(-1/p) = 1$.
Observe also that $p \e q \m{4}$, so if $p \e 1 \m{4}$, then $(q/p) = (p/q)$.
So, if $p \e 1 \m{4}$, then $(p/q) = (a/p)$, and $(a/p) = (a/q)$.
If $p \e 3 \m{4}$, then $(-1/p) = -1$ and $(q/p) = -(p/q)$.
So then $-(p/q) = -(a/p)$, or $(p/q) = (a/p)$, and $(a/p) = (a/q)$.

\end{document}
EOF
